%========================================================================
% Modelo para elaboracao de textos academicos: TCC, dissertacoes e teses
% Elaborado pelo GISIS - Grupo de Imageamento Sismico e Inversao Sismica.
%========================================================================
% ---
% PACOTES
% ---
\usepackage[brazil]{babel}
\usepackage{helvet}
\usepackage[utf8]{inputenc}		% Codificacao do documento (conversão automática dos acentos)
\usepackage{nomencl} 			% Lista de simbolos
\usepackage{color}				% Controle das cores
\usepackage{graphicx}			% Inclusão de gráficos
\usepackage{sidecap}			% mudar posição legenda
\usepackage{microtype} 			% para melhorias de justificação
\usepackage{setspace}
\usepackage{nonfloat}
\usepackage[alf,abnt-substyle=GISIS]{abntex2cite}
\usepackage{xspace}
\usepackage{amsfonts, amssymb,amsmath}
\usepackage{xspace}
\usepackage{array,booktabs}
\usepackage{marginnote}
\usepackage{comment}
\usepackage{hyperref}
\usepackage[small]{caption}
\usepackage[table]{xcolor}
\usepackage{float}
\usepackage{pdfpages}
\usepackage{enumitem}
\usepackage{multirow}
\usepackage{multicol}
\setlength{\columnsep}{1cm}
\usepackage{colortbl}
\usepackage{lipsum}
\usepackage{lastpage}
\usepackage{titlesec}
\usepackage{subfig}
\usepackage{indentfirst}		% Indenta o primeiro parágrafo de cada seção.
\usepackage[section]{placeins}  % para que as imagens fiquem na seção
\usepackage{geometry}
\usepackage{ragged2e}
% ---
% NEW COMMANDS
% ---
\renewcommand{\familydefault}{\sfdefault}

% Configuração PDF e Links clicáveis %%%%%%%%%%%%
\makeatletter
\hypersetup{
	colorlinks=true,
	linkcolor=black,
    citecolor=black,
    urlcolor=black,
}
\AtBeginDocument{
\hypersetup{
	pdftitle={\imprimirtitulo},
	pdfauthor={\imprimirautor},
	pdfsubject={\imprimirpreambulo},
	pdfcreator={LaTeX with abnTeX2},
}
}

\makeatother

\tolerance=1
\emergencystretch=\maxdimen
\hyphenpenalty=10000
\hbadness=10000

%% How many levels of section head would you like to appear in the Table of Contents?
%% 0= chapter titles, 1= section titles, 2= subsection titles, 3= subsubsection titles.
\setcounter{tocdepth}{2}

%======== Marca d'Água ================
\newcommand{\marcadaguaTemplateGISIS}{
\usepackage{draftwatermark}
\SetWatermarkAngle{45} 
\SetWatermarkLightness{.8} 
\SetWatermarkFontSize{6cm} 
\SetWatermarkScale{0.5} 
\SetWatermarkText{Template GISIS}
}
%======================================

% Tamanhos da Fonte
\newcommand{\numLarge}{\fontsize{56}{18}\selectfont}
\newcommand{\chapterLarge}{\fontsize{18}{18}\selectfont}
\newcommand{\fonteResumo}{\fontsize{12}{14}\selectfont}
\newcommand{\fonteDedicatoria}{\fontsize{11}{13}\selectfont }
\newcommand{\fonteAgradecimentos}{\fontsize{12}{12}\selectfont }
\newcommand{\fonteEpigrafe}{\fontsize{12}{13}\selectfont}
\newcommand{\fonteAutor}{\fontsize{14}{14}\selectfont}
\newcommand{\fonteTituloCapa}{\fontsize{14}{14}\selectfont}
\newcommand{\fonteTipoTrabalho}{\fontsize{14}{14}\selectfont}
\newcommand{\fontePrograma}{\fontsize{14}{14}\selectfont}
\newcommand{\fonteLocalData}{\fontsize{12}{14}\selectfont}
\newcommand{\fonteNormalSize}{\fontsize{12}{12}\selectfont}
\newcommand{\fonteTexto}{\fontsize{12}{16}\selectfont }
\newcommand{\fonteLista}{\huge }

%Outras definicoes
\makeatletter

\def\programa#1{\gdef\@programa{#1}}
\def\@programa{\@latex@warning@no@line{No \noexpand\programa given}}
\newcommand{\imprimirprograma}{\@programa}

\def\instituto#1{\gdef\@instituto{#1}}
\def\@instituto{\@latex@warning@no@line{No \noexpand\instituto given}}
\newcommand{\imprimirinstituto}{\@instituto}

\def\departamento#1{\gdef\@departamento{#1}}
\def\@departamento{\@latex@warning@no@line{No \noexpand\departamento given}}
\newcommand{\imprimirdepartamento}{\@departamento}

\def\instituicaoIngles#1{\gdef\@instituicaoIngles{#1}}
\def\@instituicaoIngles{\@latex@warning@no@line{No \noexpand\instituicaoIngles given}}
\newcommand{\imprimirinstituicaoIngles}{\@instituicaoIngles}

\def\tipotrabalhoIngles#1{\gdef\@tipotrabalhoIngles{#1}}
\def\@tipotrabalhoIngles{\@latex@warning@no@line{No \noexpand\tipotrabalhoIngles given}}
\newcommand{\imprimirtipotrabalhoIngles}{\@tipotrabalhoIngles}

\def\Sobrenome#1{\gdef\@Sobrenome{#1}}
\def\@Sobrenome{\@latex@warning@no@line{No \noexpand\Sobrenome given}}
\newcommand{\imprimirSobrenome}{\@Sobrenome}

\def\PrimeirosNomes#1{\gdef\@PrimeirosNomes{#1}}
\def\@PrimeirosNomes{\@latex@warning@no@line{No \noexpand\PrimeirosNomes given}}
\newcommand{\imprimirPrimeirosNomes}{\@PrimeirosNomes}

\def\AnoDeDefesa#1{\gdef\@AnoDeDefesa{#1}}
\def\@AnoDeDefesa{\@latex@warning@no@line{No \noexpand\AnoDeDefesa given}}
\newcommand{\imprimirAnoDeDefesa}{\@AnoDeDefesa}

\def\TituloEmIngles#1{\gdef\@TituloEmIngles{#1}}
\def\@TituloEmIngles{\@latex@warning@no@line{No \noexpand\TituloEmIngles given}}
\newcommand{\imprimirTituloEmIngles}{\@TituloEmIngles}
\makeatother

%%%%%%%%%
\geometry{
	a4paper,
	left=2.5cm,
	right=2.5cm,
	top= 2.5cm,
	bottom=2.5cm
}

\setsidefootheight{100mm}
\setlength{\parindent}{0.9cm}

% DEFINIÇÃO DAS CORES %%%%%%%%%%%%%%%%%%%%%%%%%%%
\definecolor{lightgray}{RGB}{236,236,236}
\definecolor{mediumgray}{RGB}{153,153,153}
\definecolor{black}{RGB}{0,0,0}
\definecolor{gray90}{RGB}{26,26,26}
\definecolor{darkgray}{RGB}{128,128,128}
\definecolor{gray}{RGB}{179,179,179}
\definecolor{aquamarineclaro}{RGB}{7,160,152} 
\definecolor{aquamarineescuro}{RGB}{4,142,135} 
\definecolor{TopoTabela}{RGB}{3,104,99} 
\definecolor{aquamarinebemclaro}{RGB}{235,252,251}
\definecolor{azulescuro}{RGB}{43,50,131}
\definecolor{azulmedio}{RGB}{163,204,204}
\definecolor{azulclaro}{RGB}{35,185,214}
\definecolor{azulUFF}{RGB}{0,90,171}
\definecolor{vermelho}{RGB}{183,25,24}
\definecolor{verde}{RGB}{67,165,22}

%%%%%% ESTILOS %%%%%%
%%%%%% ESTILO PARA CAPÍTULO %%%%%%
\makepagestyle{meuestiloCapitulo}
	\makeoddhead{meuestiloCapitulo} %%pagina ímpar ou com oneside
		{}
		{}
		{} 
	%% rodapé
	\makeoddfoot{meuestiloCapitulo} %%pagina ímpar ou com oneside
		{}
		{}
		{}

\makepagestyle{meuestilo}
	%%cabeçalhos
%	\makeevenhead{meuestilo} %%pagina par
%		{}
%		{}
%		{}
%	\makeoddhead{meuestilo} %%pagina ímpar ou com oneside
%		{}
%		{}
%		{}
%	\makeheadrule{meuestilo}{\textwidth}{\normalrulethickness} %linha
	

%% rodapé
\makeoddfoot{meuestilo} %%pagina ímpar ou com oneside
	{\chaptertitle}
	{}
	{\thepage}
	%\makefootrule{meuestilo}{\textwidth}{\normalrulethickness}{\footruleskip}


%%%Configuracao da Capa
\renewcommand{\imprimircapa}{
    \newgeometry{left=2.5cm, right=2.5cm, top=2.5cm, bottom=1.5cm}
    \begin{capa}
    \begin{center}
	\begin{table}[h]
        \begin{tabular}{ll}
            \multicolumn{1}{r}{\fontsize{20pt}{64pt}\selectfont} &  \multirow{4}{*}{\includegraphics[width=2.2cm]{Imgs/0_Logos/UFF_brasao.png}} \\
            \multicolumn{1}{r}{\MakeUppercase{\fontsize{14pt}{64pt}\selectfont\imprimirinstituicao}} &\\[1ex]
            \multicolumn{1}{r}{\MakeUppercase{\fontsize{14pt}{64pt}\selectfont \imprimirinstituto}} &\\[1ex]
            
            \multicolumn{1}{r}{\MakeUppercase{\fontsize{14pt}{64pt}\selectfont \imprimirdepartamento}} & 
        \end{tabular}
    \end{table}
    \vspace{4cm}
    \MakeUppercase{\fonteAutor \imprimirautor} 
    
    \vspace{3cm}
	\MakeUppercase{\fonteTituloCapa \textbf{\imprimirtitulo}} 
	
	%\vspace{6cm}
	\vfill
	\MakeUppercase{\fonteTipoTrabalho \imprimirtipotrabalho}
	
	\vspace{2cm}
	\MakeUppercase{\fontePrograma \imprimirprograma }
	
	\vspace{3cm}
	\fonteLocalData \textbf{\imprimirlocal \\ \imprimirdata}
	
	\end{center}
    
    \end{capa}
    \restoregeometry
}

%%%Configuracao da Folha de Rosto
\makeatletter
\renewcommand{\imprimirfolhaderosto}{
    \newgeometry{left=2.5cm, right=2.5cm, top=2.5cm, bottom=1.5cm}
    \begin{center}
        \MakeUppercase{\fonteAutor \imprimirautor}
        
        \vspace{5.3cm}
        \MakeUppercase{\fonteTituloCapa \textbf{\imprimirtitulo}} 
        
        \vspace{1cm}
        \abntex@ifnotempty{\imprimirpreambulo}{%
            \hspace{.45\textwidth}
            \begin{minipage}{.5\textwidth}
            \small
            \SingleSpacing
            \imprimirpreambulo
            \end{minipage}%
            \vspace*{\fill}
        }%
        
        \abntex@ifnotempty{\imprimirorientadorRotulo}{%
            \hspace{.45\textwidth}
            \begin{minipage}{.5\textwidth}
            \fonteNormalSize \imprimirorientadorRotulo\\
            \imprimirorientador
            
            \vspace{0.5cm}
            \imprimircoorientadorRotulo\\
            \imprimircoorientador
            \end{minipage}%
            \vspace*{\fill}
        }%
        
        \vfill
	\fonteLocalData \textbf{\imprimirlocal \\ \imprimirdata}
    \end{center}
    \restoregeometry
    }
\makeatother

%%%Configuracao da Ficha Catalografica
\newcommand{\imprimirfichacatalografica}{

\newpage
\begin{fichacatalografica}
	\vspace*{\fill}					% Posição vertical
	\hrule							% Linha horizontal
	\begin{center}					% Minipage Centralizado
	{\huge \color{red} \textbf{Inserir a ficha catalografica gerada pelo SDC/BIG da UFF}}
	
	\end{center}
	\hrule
\end{fichacatalografica}
}

%================== Capa - Template =======================
\newcommand{\imprimircapatemplate}{

\newpage

\begin{center}
\includegraphics[width=4cm]{Imgs/0_Logos/Logo_GISIS_topo_v2.png} \\
\vspace{.5cm}
\textbf{\large Template do GISIS para monografias (TCC), dissertações de mestrado e teses de doutorado}    

\vfill

\MakeUppercase{Autoras:}
\begin{itemize}
    \item Geofísica Bruna Carbonesi
    \item Dra. Danielle Martins Tostes
    \item Stephanie Tavares
\end{itemize}

\MakeUppercase{Revisores:}

\begin{itemize}
    \item Dr. Alexandre Maul
    \item Dr. Felipe Timóteo
    \item Prof. Dr. Luiz Alberto Santos
    \item Prof. Dr. Marco Cetale
\end{itemize}

\vfill

\textbf{\large Venha fazer parte do GISIS!} 

\vfill

\MakeUppercase{Professores:}
\begin{itemize}
    \item Dr. Alexandre Motta
    \item Dr. Leonardo Miquelutti
    \item Dr. Luiz Alberto Santos
    \item Dr. Marco Cetale
    \item Dr. Roger Matsumoto
\end{itemize}

\end{center}
\newpage
%=====================================
}
