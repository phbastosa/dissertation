%========================================================================
% Modelo para elaboracao de textos academicos: TCC, dissertacoes e teses
% Elaborado pelo GISIS - Grupo de Imageamento Sismico e Inversao Sismica.
%========================================================================
% ----------------------------------------------------------
\chapter{Informações sobre anexos}
\label{an:titulo}
% ----------------------------------------------------------

O anexo, assim como o apêndice, é um elemento pós-textual que faz parte dos trabalhos de pesquisa. Ele também é utilizado para complementar ou comprovar a argumentação e pode ser um texto ou documento, que não foi elaborado pelo autor.

Segundo a ABNT (Associação Brasileira de Normas Técnicas), a principal diferença entre anexo e apêndice é que os apêndices são textos criados pelo próprio autor para complementar sua argumentação, enquanto os anexos são documentos criados por terceiros, e usados pelo autor.

Os anexos vem depois dos apêndices. Exemplos comuns de anexos são mapas, estatutos, leis, imagens, etc.