% ========================================================================
% Modelo para elaboracao de textos academicos: TCC, dissertacoes e teses
% Elaborado pelo GISIS - Grupo de Imageamento Sismico e Inversao Sismica.
% ========================================================================
%Autoras:
%   Bruna Carbonesi
%   Danielle Martins Tostes
%   Stephanie Tavares
%
%Revisores:
%   Prof. Marco Cetale
%   Prof. Luiz Alberto Santos
%   Alexandre Maul
%   Felipe Timoteo da Costa
%
%Ultima atualizacao: 18 de abril de 2023
%============================================================

\documentclass[a4paper, 12pt, oneside]{abntex2}

%========================================================================
% Modelo para elaboracao de textos academicos: TCC, dissertacoes e teses
% Elaborado pelo GISIS - Grupo de Imageamento Sismico e Inversao Sismica.
%========================================================================
% ---
% PACOTES
% ---
\usepackage[brazil]{babel}
\usepackage{helvet}
\usepackage[utf8]{inputenc}		% Codificacao do documento (conversão automática dos acentos)
\usepackage{nomencl} 			% Lista de simbolos
\usepackage{color}				% Controle das cores
\usepackage{graphicx}			% Inclusão de gráficos
\usepackage{sidecap}			% mudar posição legenda
\usepackage{microtype} 			% para melhorias de justificação
\usepackage{setspace}
\usepackage{nonfloat}
\usepackage[alf,abnt-substyle=GISIS]{abntex2cite}
\usepackage{xspace}
\usepackage{amsfonts, amssymb,amsmath}
\usepackage{xspace}
\usepackage{array,booktabs}
\usepackage{marginnote}
\usepackage{comment}
\usepackage{hyperref}
\usepackage[small]{caption}
\usepackage[table]{xcolor}
\usepackage{float}
\usepackage{pdfpages}
\usepackage{enumitem}
\usepackage{multirow}
\usepackage{multicol}
\setlength{\columnsep}{1cm}
\usepackage{colortbl}
\usepackage{lipsum}
\usepackage{lastpage}
\usepackage{titlesec}
\usepackage{subfig}
\usepackage{indentfirst}		% Indenta o primeiro parágrafo de cada seção.
\usepackage[section]{placeins}  % para que as imagens fiquem na seção
\usepackage{geometry}
\usepackage{ragged2e}
% ---
% NEW COMMANDS
% ---
\renewcommand{\familydefault}{\sfdefault}

% Configuração PDF e Links clicáveis %%%%%%%%%%%%
\makeatletter
\hypersetup{
	colorlinks=true,
	linkcolor=black,
    citecolor=black,
    urlcolor=black,
}
\AtBeginDocument{
\hypersetup{
	pdftitle={\imprimirtitulo},
	pdfauthor={\imprimirautor},
	pdfsubject={\imprimirpreambulo},
	pdfcreator={LaTeX with abnTeX2},
}
}

\makeatother

\tolerance=1
\emergencystretch=\maxdimen
\hyphenpenalty=10000
\hbadness=10000

%% How many levels of section head would you like to appear in the Table of Contents?
%% 0= chapter titles, 1= section titles, 2= subsection titles, 3= subsubsection titles.
\setcounter{tocdepth}{2}

%======== Marca d'Água ================
\newcommand{\marcadaguaTemplateGISIS}{
\usepackage{draftwatermark}
\SetWatermarkAngle{45} 
\SetWatermarkLightness{.8} 
\SetWatermarkFontSize{6cm} 
\SetWatermarkScale{0.5} 
\SetWatermarkText{Template GISIS}
}
%======================================

% Tamanhos da Fonte
\newcommand{\numLarge}{\fontsize{56}{18}\selectfont}
\newcommand{\chapterLarge}{\fontsize{18}{18}\selectfont}
\newcommand{\fonteResumo}{\fontsize{12}{14}\selectfont}
\newcommand{\fonteDedicatoria}{\fontsize{11}{13}\selectfont }
\newcommand{\fonteAgradecimentos}{\fontsize{12}{12}\selectfont }
\newcommand{\fonteEpigrafe}{\fontsize{12}{13}\selectfont}
\newcommand{\fonteAutor}{\fontsize{14}{14}\selectfont}
\newcommand{\fonteTituloCapa}{\fontsize{14}{14}\selectfont}
\newcommand{\fonteTipoTrabalho}{\fontsize{14}{14}\selectfont}
\newcommand{\fontePrograma}{\fontsize{14}{14}\selectfont}
\newcommand{\fonteLocalData}{\fontsize{12}{14}\selectfont}
\newcommand{\fonteNormalSize}{\fontsize{12}{12}\selectfont}
\newcommand{\fonteTexto}{\fontsize{12}{16}\selectfont }
\newcommand{\fonteLista}{\huge }

%Outras definicoes
\makeatletter

\def\programa#1{\gdef\@programa{#1}}
\def\@programa{\@latex@warning@no@line{No \noexpand\programa given}}
\newcommand{\imprimirprograma}{\@programa}

\def\instituto#1{\gdef\@instituto{#1}}
\def\@instituto{\@latex@warning@no@line{No \noexpand\instituto given}}
\newcommand{\imprimirinstituto}{\@instituto}

\def\departamento#1{\gdef\@departamento{#1}}
\def\@departamento{\@latex@warning@no@line{No \noexpand\departamento given}}
\newcommand{\imprimirdepartamento}{\@departamento}

\def\instituicaoIngles#1{\gdef\@instituicaoIngles{#1}}
\def\@instituicaoIngles{\@latex@warning@no@line{No \noexpand\instituicaoIngles given}}
\newcommand{\imprimirinstituicaoIngles}{\@instituicaoIngles}

\def\tipotrabalhoIngles#1{\gdef\@tipotrabalhoIngles{#1}}
\def\@tipotrabalhoIngles{\@latex@warning@no@line{No \noexpand\tipotrabalhoIngles given}}
\newcommand{\imprimirtipotrabalhoIngles}{\@tipotrabalhoIngles}

\def\Sobrenome#1{\gdef\@Sobrenome{#1}}
\def\@Sobrenome{\@latex@warning@no@line{No \noexpand\Sobrenome given}}
\newcommand{\imprimirSobrenome}{\@Sobrenome}

\def\PrimeirosNomes#1{\gdef\@PrimeirosNomes{#1}}
\def\@PrimeirosNomes{\@latex@warning@no@line{No \noexpand\PrimeirosNomes given}}
\newcommand{\imprimirPrimeirosNomes}{\@PrimeirosNomes}

\def\AnoDeDefesa#1{\gdef\@AnoDeDefesa{#1}}
\def\@AnoDeDefesa{\@latex@warning@no@line{No \noexpand\AnoDeDefesa given}}
\newcommand{\imprimirAnoDeDefesa}{\@AnoDeDefesa}

\def\TituloEmIngles#1{\gdef\@TituloEmIngles{#1}}
\def\@TituloEmIngles{\@latex@warning@no@line{No \noexpand\TituloEmIngles given}}
\newcommand{\imprimirTituloEmIngles}{\@TituloEmIngles}
\makeatother

%%%%%%%%%
\geometry{
	a4paper,
	left=2.5cm,
	right=2.5cm,
	top= 2.5cm,
	bottom=2.5cm
}

\setsidefootheight{100mm}
\setlength{\parindent}{0.9cm}

% DEFINIÇÃO DAS CORES %%%%%%%%%%%%%%%%%%%%%%%%%%%
\definecolor{lightgray}{RGB}{236,236,236}
\definecolor{mediumgray}{RGB}{153,153,153}
\definecolor{black}{RGB}{0,0,0}
\definecolor{gray90}{RGB}{26,26,26}
\definecolor{darkgray}{RGB}{128,128,128}
\definecolor{gray}{RGB}{179,179,179}
\definecolor{aquamarineclaro}{RGB}{7,160,152} 
\definecolor{aquamarineescuro}{RGB}{4,142,135} 
\definecolor{TopoTabela}{RGB}{3,104,99} 
\definecolor{aquamarinebemclaro}{RGB}{235,252,251}
\definecolor{azulescuro}{RGB}{43,50,131}
\definecolor{azulmedio}{RGB}{163,204,204}
\definecolor{azulclaro}{RGB}{35,185,214}
\definecolor{azulUFF}{RGB}{0,90,171}
\definecolor{vermelho}{RGB}{183,25,24}
\definecolor{verde}{RGB}{67,165,22}

%%%%%% ESTILOS %%%%%%
%%%%%% ESTILO PARA CAPÍTULO %%%%%%
\makepagestyle{meuestiloCapitulo}
	\makeoddhead{meuestiloCapitulo} %%pagina ímpar ou com oneside
		{}
		{}
		{} 
	%% rodapé
	\makeoddfoot{meuestiloCapitulo} %%pagina ímpar ou com oneside
		{}
		{}
		{}

\makepagestyle{meuestilo}
	%%cabeçalhos
%	\makeevenhead{meuestilo} %%pagina par
%		{}
%		{}
%		{}
%	\makeoddhead{meuestilo} %%pagina ímpar ou com oneside
%		{}
%		{}
%		{}
%	\makeheadrule{meuestilo}{\textwidth}{\normalrulethickness} %linha
	

%% rodapé
\makeoddfoot{meuestilo} %%pagina ímpar ou com oneside
	{\chaptertitle}
	{}
	{\thepage}
	%\makefootrule{meuestilo}{\textwidth}{\normalrulethickness}{\footruleskip}


%%%Configuracao da Capa
\renewcommand{\imprimircapa}{
    \newgeometry{left=2.5cm, right=2.5cm, top=2.5cm, bottom=1.5cm}
    \begin{capa}
    \begin{center}
	\begin{table}[h]
        \begin{tabular}{ll}
            \multicolumn{1}{r}{\fontsize{20pt}{64pt}\selectfont} &  \multirow{4}{*}{\includegraphics[width=2.2cm]{Imgs/0_Logos/UFF_brasao.png}} \\
            \multicolumn{1}{r}{\MakeUppercase{\fontsize{14pt}{64pt}\selectfont\imprimirinstituicao}} &\\[1ex]
            \multicolumn{1}{r}{\MakeUppercase{\fontsize{14pt}{64pt}\selectfont \imprimirinstituto}} &\\[1ex]
            
            \multicolumn{1}{r}{\MakeUppercase{\fontsize{14pt}{64pt}\selectfont \imprimirdepartamento}} & 
        \end{tabular}
    \end{table}
    \vspace{4cm}
    \MakeUppercase{\fonteAutor \imprimirautor} 
    
    \vspace{3cm}
	\MakeUppercase{\fonteTituloCapa \textbf{\imprimirtitulo}} 
	
	%\vspace{6cm}
	\vfill
	\MakeUppercase{\fonteTipoTrabalho \imprimirtipotrabalho}
	
	\vspace{2cm}
	\MakeUppercase{\fontePrograma \imprimirprograma }
	
	\vspace{3cm}
	\fonteLocalData \textbf{\imprimirlocal \\ \imprimirdata}
	
	\end{center}
    
    \end{capa}
    \restoregeometry
}

%%%Configuracao da Folha de Rosto
\makeatletter
\renewcommand{\imprimirfolhaderosto}{
    \newgeometry{left=2.5cm, right=2.5cm, top=2.5cm, bottom=1.5cm}
    \begin{center}
        \MakeUppercase{\fonteAutor \imprimirautor}
        
        \vspace{5.3cm}
        \MakeUppercase{\fonteTituloCapa \textbf{\imprimirtitulo}} 
        
        \vspace{1cm}
        \abntex@ifnotempty{\imprimirpreambulo}{%
            \hspace{.45\textwidth}
            \begin{minipage}{.5\textwidth}
            \small
            \SingleSpacing
            \imprimirpreambulo
            \end{minipage}%
            \vspace*{\fill}
        }%
        
        \abntex@ifnotempty{\imprimirorientadorRotulo}{%
            \hspace{.45\textwidth}
            \begin{minipage}{.5\textwidth}
            \fonteNormalSize \imprimirorientadorRotulo\\
            \imprimirorientador
            
            \vspace{0.5cm}
            \imprimircoorientadorRotulo\\
            \imprimircoorientador
            \end{minipage}%
            \vspace*{\fill}
        }%
        
        \vfill
	\fonteLocalData \textbf{\imprimirlocal \\ \imprimirdata}
    \end{center}
    \restoregeometry
    }
\makeatother

%%%Configuracao da Ficha Catalografica
\newcommand{\imprimirfichacatalografica}{

\newpage
\begin{fichacatalografica}
	\vspace*{\fill}					% Posição vertical
	\hrule							% Linha horizontal
	\begin{center}					% Minipage Centralizado
	{\huge \color{red} \textbf{Inserir a ficha catalografica gerada pelo SDC/BIG da UFF}}
	
	\end{center}
	\hrule
\end{fichacatalografica}
}

%================== Capa - Template =======================
\newcommand{\imprimircapatemplate}{

\newpage

\begin{center}
\includegraphics[width=4cm]{Imgs/0_Logos/Logo_GISIS_topo_v2.png} \\
\vspace{.5cm}
\textbf{\large Template do GISIS para monografias (TCC), dissertações de mestrado e teses de doutorado}    

\vfill

\MakeUppercase{Autoras:}
\begin{itemize}
    \item Geofísica Bruna Carbonesi
    \item Dra. Danielle Martins Tostes
    \item Stephanie Tavares
\end{itemize}

\MakeUppercase{Revisores:}

\begin{itemize}
    \item Dr. Alexandre Maul
    \item Dr. Felipe Timóteo
    \item Prof. Dr. Luiz Alberto Santos
    \item Prof. Dr. Marco Cetale
\end{itemize}

\vfill

\textbf{\large Venha fazer parte do GISIS!} 

\vfill

\MakeUppercase{Professores:}
\begin{itemize}
    \item Dr. Alexandre Motta
    \item Dr. Leonardo Miquelutti
    \item Dr. Luiz Alberto Santos
    \item Dr. Marco Cetale
    \item Dr. Roger Matsumoto
\end{itemize}

\end{center}
\newpage
%=====================================
}

%========================================================================
% Modelo para elaboracao de textos academicos: TCC, dissertacoes e teses
% Elaborado pelo GISIS - Grupo de Imageamento Sismico e Inversao Sismica.
%========================================================================
%%%Configuracao da Folha de Aprovacao
\newcommand{\imprimirfolhadeaprovacao}{
    \newpage
    \newgeometry{left=2.5cm, right=2.5cm, top=2.5cm, bottom=1.5cm}
    \begin{center}
        \MakeUppercase{\fonteAutor \imprimirautor}
        
        \vspace{0.5cm}
        \MakeUppercase{\fonteNormalSize \textbf{\imprimirtitulo}} 
        
        \vspace{1cm}
            \hspace{.45\textwidth}
            \begin{minipage}{.5\textwidth}
	    \small
            \SingleSpacing
            \imprimirpreambulo
            
            \end{minipage}
            \vspace{2cm}
        
            \fontsize{11}{10}\selectfont Aprovada em 25 de Setembro de 2023 pela banca examinadora abaixo:
            
            \vspace{1cm}
            \noindent\makebox[\linewidth]{\rule{10cm}{0.5pt}} 
            \textbf{\imprimirorientador\ (Orientador)}\\
            Universidade Federal Fluminense
            
            \vspace{1cm}
            \noindent\makebox[\linewidth]{\rule{10cm}{0.5pt}} 
            \textbf{\imprimircoorientador\ (Coorientador)}\\
            Universidade Federal Fluminense / Petrobras
            
            \vspace{1cm}
            \noindent\makebox[\linewidth]{\rule{10cm}{0.5pt}} 
            \textbf{Prof. Dr. Jessé Carvalho Costa}\\
            Universidade Federal do Pará
            
            \vspace{1cm}
            \noindent\makebox[\linewidth]{\rule{10cm}{0.5pt}} 
            \textbf{Prof. Dr. Marcelo Peres Rocha}\\
            Universidade de Brasília
            
            \vspace{1cm}
            \noindent\makebox[\linewidth]{\rule{10cm}{0.5pt}} 
            \textbf{Prof. Dr. Roger Matsumoto Moreira}\\
            Universidade Federal Fluminense
        
        \vfill
	\fonteLocalData \textbf{\imprimirlocal \\ \imprimirdata}
    \end{center}
    \restoregeometry
}
 %Neste arquivo devem ser preechidos os dados dos membros da banca

%=================================================
%         Informacoes do Trabalho
%=================================================
\autor{Paulo Henrique Bastos Alves}
\Sobrenome{Alves}
\PrimeirosNomes{Paulo Henrique Bastos}
\titulo{Análise da acurácia dos tempos de trânsito obtidos via modelagem 3D com equação eikonal}
\TituloEmIngles{Travel time accuracy analysis obtained by 3D eikonal equation modeling}
\instituicao{Universidade Federal Fluminense}
\instituicaoIngles{Fluminense Federal University}
\instituto{Instituto de Geociências}
\departamento{Departamento de Geologia e Geofísica}
\orientador[\textbf{Orientador}]{Prof. Dr. Marco Antonio Cetale Santos}
\coorientador[\textbf{Coorientador}]{Prof. Dr. Luiz Alberto Santos} 
\local{Niterói - RJ}
\data{Setembro / 2023}
\AnoDeDefesa{2023}

% Escolha e descomente abaixo a configuracao adequada ao tipo do seu trabalho

%=================== Monografia ========================
% \preambulo{Monografia apresentada à Universidade Federal Fluminense como requisito parcial do Curso de Graduação em Geofísica para a obtenção do título de Bacharel em Geofísica.}
% \tipotrabalho{Monografia}
% \tipotrabalhoIngles{Monography}
% \programa{Curso de Graduação em Geofísica}

%============ Dissertacao de Mestrado ==================
\preambulo{Dissertação apresentada à Universidade Federal Fluminense como requisito parcial do Programa de Pós-Graduação em Dinâmica dos Oceanos e da Terra para a obtenção do título de Mestre em Ciências. \\ \vspace{0.2cm} \\Área de concentração: Geologia e Geofísica.}
\tipotrabalho{Dissertação de Mestrado}
\tipotrabalhoIngles{Dissertation (MSc.: Master of Science)}
\programa{Programa de Pós-Graduação \\ Dinâmica dos Oceanos e da Terra (PPGDOT)}

%============== Tese de Doutorado ======================
%\preambulo{Tese apresentada à Universidade Federal Fluminense como requisito parcial do Programa de Pós-Graduação em Dinâmica dos Oceanos e da Terra para a obtenção do título de Doutor em Ciências. \\ \vspace{0.2cm} \\Área de concentração: Geologia e Geofísica.}
%\tipotrabalho{Tese de Doutorado}
%\tipotrabalhoIngles{Thesis (PhD.: Philosophy Doctor)}
%\programa{Programa de Pós-Graduação \\ Dinâmica dos Oceanos e da Terra (DOT)}

\begin{document}

% CAPA %%%%%%%%%%%%%%%%%%%%%%%
\imprimircapa

% FOLHA DE ROSTO %%%%%%%%%%%%%
\imprimirfolhaderosto

% FICHA CATALOGRÁFICA %%%%%%%% 
%\imprimirfichacatalografica
\includepdf{ficha_catalografica.pdf} %opção para incluir a ficha gerada

% FOLHA DE APROVAÇÃO %%%%%%%%
%as informacoes sobre a banca devem ser incluidas no arquivo config_folhadeaprovacao.tex
\imprimirfolhadeaprovacao
%\includepdf{folhadeaprovacao.pdf} %opção para incluir a folha assinada

\setlength{\parskip}{0.5cm}	% Controle do espaçamento entre parágrafos
% Dedicatória %%%%%%%%%%%%%%%% 
%========================================================================
% Modelo para elaboracao de textos academicos: TCC, dissertacoes e teses
% Elaborado pelo GISIS - Grupo de Imageamento Sismico e Inversao Sismica.
%========================================================================
\newpage
    \begin{dedicatoria}
       \vspace*{\fill}
       \centering
       \hspace{.9\textwidth}
            \begin{minipage}{.6\textwidth}
            \setlength{\parskip}{0.5cm}	
            \fonteDedicatoria 
            \begin{itshape}
            	Eu dedico esse trabalho a todos os estudantes de baixa renda, cuja força e determinação transcendem os desafios que enfrentam diariamente, provando que a adversidade pode ser superada com coragem e perseverança. Mesmo diante das limitações impostas pela falta de recursos, vocês encontram uma maneira de brilhar, buscando o conhecimento e abraçando o poder transformador da educação. À medida que prosseguem em suas jornadas acadêmicas, que esta dedicatória sirva como um lembrete constante de seu valor e potencial ilimitado. Vocês são a prova de que a resiliência e a paixão pelo conhecimento podem romper as correntes da desigualdade.        
            \end{itshape}
            \end{minipage}
        \vspace*{\fill}
    \end{dedicatoria}

% Agradecimentos %%%%%%%%%%%%%% 
%========================================================================
% Modelo para elaboracao de textos academicos: TCC, dissertacoes e teses
% Elaborado pelo GISIS - Grupo de Imageamento Sismico e Inversao Sismica.
%========================================================================
\begin{agradecimentos}
    \fonteAgradecimentos 
    
	Eu gostaria de agradecer à Universidade Federal Fluminense pelo espaço concedido para a pesquisa, pela infraestrutura fornecida e pela iniciativa de eventos culturais e sociais implantados. 
	
	Obrigado ao Programa de Pós-Graduação em Dinâmica dos Oceanos e da Terra (PPGDOT) pela dedicação no processo seletivo e administração das disciplinas oferecidas. 
	
	Obrigado ao Grupo de Imageamento Sísmico e Inversão Sísmica (GISIS) pelas ideias compartilhadas, pelas vivências em grupo e por toda infraestrutura oferecida para o desenvolvimento deste trabalho. 
	
	Obrigado ao projeto de pesquisa e desenvolvimento "sísmica de refração para monitoramento de reservatório do pré-sal", gerenciado pela empresa SHELL Brasil em parceria com a Agência Nacional do Petróleo (ANP nº 21727-3) que financiou a minha bolsa de mestrado. 
	
	Obrigado aos meus orientadores pelas conversas e ajudas ao longo da minha jornada acadêmica. 
	
	Obrigado aos membros externos e internos da banca que avaliaram este trabalho com zelo e dedicação. 
	
	Agradeço a minha família pelo apoio financeiro no início da minha jornada do mestrado e concluindo com o apoio emocional para o término da minha pesquisa. 
	
	Agradeço às pessoas que mantiveram contato comigo desde o início até a finalização da minha jornada, que me possibilitaram ter bem estar social durante o desenvolvimento dos experimentos até a escrita e apresentação deste trabalho.         
    
\end{agradecimentos}

% Epígrafe %%%%%%%%%%%%%%%%%%%% 
%========================================================================
% Modelo para elaboracao de textos academicos: TCC, dissertacoes e teses
% Elaborado pelo GISIS - Grupo de Imageamento Sismico e Inversao Sismica.
%========================================================================
\begin{epigrafe}
        \vspace*{\fill}
    	\hspace{\fill}
            \begin{minipage}{.5\textwidth}
            \fonteEpigrafe \begin{flushright}
    		\textbf{
    		\textit{
    		[ITEM OPCIONAL]\\
    		Epígrafe é uma citação de provérbio, pensamento ou frase que está relacionada ao mote do trabalho.\\  
    		“O único lugar onde o sucesso vem antes do trabalho é no dicionário.”\\
    		Autor Desconhecido
    		}}
    	    \end{flushright}
            \end{minipage}
\end{epigrafe}

% Resumos %%%%%%%%%%%%%%%%%%%%% 
%========================================================================
% Modelo para elaboracao de textos academicos: TCC, dissertacoes e teses
% Elaborado pelo GISIS - Grupo de Imageamento Sismico e Inversao Sismica.
%========================================================================
\setlength{\absparsep}{18pt} % ajusta o espaçamento dos parágrafos do resumo
\begin{resumo}
    \fonteResumo

\begin{flushleft}
\MakeUppercase{\imprimirSobrenome}, \imprimirPrimeirosNomes. \textbf{\imprimirtitulo}. \imprimirtipotrabalho, \imprimirinstituicao. \imprimirlocal,  p. \pageref{LastPage}. \imprimirAnoDeDefesa.
\end{flushleft}
	
	texto aqui
	
    \textbf{Palavras-chave}: equação eikonal; sísmica de refração; comparação analítica-numérica; computação de alto desempenho.
\end{resumo}

%========================================================================
% Modelo para elaboracao de textos academicos: TCC, dissertacoes e teses
% Elaborado pelo GISIS - Grupo de Imageamento Sismico e Inversao Sismica.
%========================================================================
\begin{resumo}[Abstract]
    \begin{otherlanguage*}{english}

	\begin{flushleft}
	\MakeUppercase{\imprimirSobrenome}, \imprimirPrimeirosNomes. \textbf{\imprimirTituloEmIngles}. \imprimirtipotrabalhoIngles, \imprimirinstituicaoIngles. \imprimirlocal,  p. \pageref{LastPage}. \imprimirAnoDeDefesa.
	\end{flushleft}

    \fonteResumo
	This study evaluates the accuracy and computational efficiency of three distinct methods for solving the eikonal equation, which represents a high-frequency approximation of the wave equation in acoustic media with constant density. The analyzed numerical methods consist, firstly, of a well-established classical formulation in the literature; secondly, a method with limited exploration in the realm of geosciences, albeit with significant computational performance; and finally, the innovation developed in this study, the combination of two formulations to generate a precise and efficient computational approach. The implementation of these formulations, using parallel programming on graphics processing units, required rigorous validation of solution accuracy to ensure the viability of the numerical methods in complex three-dimensional contexts. To verify the accuracy of the solutions, three strategies were adopted, including the application of the methods in a homogeneous model, in which a recent formulation was tested in a restricted manner, the evaluation in a two-layer model with variations in size, allowing analysis of algorithm scalability, and finally, the analysis in a realistic model that incorporates complex geology, providing detailed results on computational efficiency. Among all the examined formulations, the Fast Sweeping Method stood out for its remarkable accuracy in calculating travel times, driven by the use of accurate operators parallelized on graphics processing units.
    
    \vspace{\onelineskip}
     
    \noindent 
    \textbf{Keywords}: eikonal equation; seismic refraction; analytical-numerical comparison; high performance computing.
    
    \end{otherlanguage*}
\end{resumo}


% Listas de figuras, tabelas, abreviaturas, siglas e símbolos
%========================================================================
% Modelo para elaboracao de textos academicos: TCC, dissertacoes e teses
% Elaborado pelo GISIS - Grupo de Imageamento Sismico e Inversao Sismica.
%========================================================================
% Lista de ilustrações %%%%%%%%
\pdfbookmark[0]{\listfigurename}{lof}
\listoffigures*
\cleardoublepage

% Lista de tabelas %%%%%%%%%%%%
\pdfbookmark[0]{\listtablename}{lot}
\listoftables*
\cleardoublepage

% Lista de abreviaturas e siglas
\begin{center}
\fonteLista Lista de abreviaturas e siglas 
\end{center}

\vspace{2cm}
\fonteNormalSize
\begin{tabular}{l l}
    UFF & Universidade Federal Fluminense \\ [1.0ex]
    GGO & Departamentode Geologia e Geofísica \\ [1.0ex]
    GISIS & Grupo de Imageamento e Inversão Sísmica \\ [1.0ex]
	WSL & \textit{Windows Subsystem for Linux} \\ [1.0ex]
	FSM & \textit{Fast Sweeping Method} \\ [1.0ex]
	FIM & \textit{Fast Iterative Method} \\ [1.0ex]
	OBN & \textit{Ocean Bottom Nodes}
\end{tabular}
\cleardoublepage


% Lista de símbolos
\begin{center}
    \fonteLista Lista de símbolos
\end{center}

\vspace{2cm}
\fonteNormalSize
\begin{tabular}{l l}
    $ \Gamma $ & Letra grega Gama \\ [1.0ex]
    $ \Lambda $ & Lambda \\ [1.0ex]
    $ \zeta $ & Letra grega minúscula zeta \\ [1.0ex]
    $ \in $ & Pertence
\end{tabular}

    


% Sumário %%%%%%%%%%%%%%%%%%%%%
\pdfbookmark[0]{\contentsname}{toc}
\tableofcontents*
\thispagestyle{empty}

\textual

\justify \fonteTexto
%========================================================================
% Modelo para elaboracao de textos academicos: TCC, dissertacoes e teses
% Elaborado pelo GISIS - Grupo de Imageamento Sismico e Inversao Sismica.
%========================================================================
\chapter{Introdução}
\label{ch:introducao}


%========================================================================
% Modelo para elaboracao de textos academicos: TCC, dissertacoes e teses
% Elaborado pelo GISIS - Grupo de Imageamento Sismico e Inversão Sismica.
%========================================================================
\chapter{Revisão Bibliográfica}
\label{ch:revisaobibliografica}

A modelagem e tomografia sísmica são técnicas essenciais na exploração de recursos naturais subterrâneos e na caracterização da subsuperfície terrestre. A tomografia sísmica é um método não invasivo que permite mapear as propriedades físicas do subsolo através da análise das ondas sísmicas geradas por fontes controladas. Um dos principais desafios na tomografia sísmica é a obtenção de informações precisas sobre as trajetórias das ondas sísmicas e suas velocidades de propagação em diferentes camadas geológicas. Nesse sentido, a equação \textit{eikonal}  é uma ferramenta fundamental para a modelagem e inversão e imageamento sísmico, pois descreve as frentes de onda e permite calcular os tempos de percurso das ondas sísmicas. Neste capítulo é explorado a equação eikonal, três resoluções numéricas e sua aplicação na tomografia de refração. Ao explorar esses tópicos, este capítulo fornecerá abrangentes conceitos utilizados na geração dos resultados deste trabalho.

\section{Modelagem sísmica}

Muitos fenômenos físicos tem suas simplificações e condições para serem simulados computacionalmente. O método sísmico parte do princípio da propagação de ondas mecânicas geradas a partir de uma fonte explosiva, sendo registradas em receptores posicionados na superfície da terra ou no fundo marinho \cite{sheriff1995exploration, rosa2010analise}. Um modelo de propriedades físicas de subsuperfície é necessário para a simulação computacional. Considerando a simplificação da equação da onda para meios acústicos e isotrópicos, onde a propriedade do meio são as velocidades da onda P, as frentes de onda podem ser geradas para realizar estudos sintéticos experimentais. Existem formatos de solução para uma equação que rege um fenômeno físico, dois deles são o caso analítica e o caso numérico. A solução analítica de um problema mostra a resposta exata do fenômeno em condições específicas previamente estabelecidas e a solução numérica resolve o problema de forma geral para diversos cenários. No caso da modelagem sísmica, as condições estabelecidas se aplicam ao modelo de velocidade, resolvendo a equação em um modelo homogêneo ou com camadas plano paralelas sem variação lateral de velocidade.  

\subsection*{Equação da onda para meios acústicos}

A equação base para a simulação sísmica utilizando a simplificação para meios acústicos pode ser formulada da seguinte maneira
\begin{equation}
	\nabla^2u(\mathbf{x}, t) - \dfrac{1}{v^2(\mathbf{x})}\dfrac{\partial^2u(\mathbf{x}, t)}{\partial t^2} = f(t),	
	\label{wave_equation}
\end{equation}
\noindent onde $u(\mathbf{x},t)$ é o campo de pressão hidrostática, $t$ é o tempo de propagação e $\mathbf{x} = (x,y,z)$ são as variáveis do sistema de coordenadas para o caso 3D. $v(\mathbf{x})$ é o modelo de velocidade, $f(t)$ é uma função que determina o formato do pulso propagado e $\nabla^2 = \partial_x + \partial_y + \partial_z$ é o operador laplaciano. \citeonline{igel2017computational} mostra algumas formas de resolver a equação da onda numericamente de forma prática e aplicada. O método das diferenças finitas é um dos métodos mais utilizados em simulações sísmicas. O princípio do método é estimar o valor da derivada numericamente utilizando a série de Taylor, definida detalhadamente no trabalho de \citeonline{de2005funccoes}. Somente derivadas de segunda ordem são necessárias para resolver a equação \ref{wave_equation} e a aproximação da derivada pode ser definida a partir da seguinte expressão
\begin{equation}
	\dfrac{\partial^2 f(x)}{\partial x^2} \approx \dfrac{f[i - dh] - 2f[i] + f[i + dh]}{dh^2},
	\label{derivada_2}
\end{equation}    
\noindent onde $f(x)$ é uma função contínua arbitrária definida no eixo $x \in \mathbb{R}$ e $f[i]$ é uma função discreta definida na mesma reta com intervalos regulares $dh$. A equação da onda para meios acústicos pode ser escrita em seu formato discreto  substituindo a equação \ref{derivada_2} na equação \ref{wave_equation}, para cada eixo $x,y,z,t$ respectivamente. O método das diferenças finitas possui certas limitações como por exemplo a dispersão e instabilidade numérica \cite{aki1980quantitative}. Seguindo os trabalhos de \citeonline{mufti1990large, bulcao2004modelagem} e \citeonline{robertsson2012numerical} as condições para contornar problemas numéricos na modelagem sísmica para meios acústicos seguem na forma a seguir
\begin{equation}
	\begin{cases}
		dh \le v_{\text{min}} \,/\, (\alpha \cdot f_{\text{max}}) \\
		dt \le dh \,/\, (\beta \cdot v_{\text{max}})
	\end{cases}
\end{equation}   
\noindent onde $dh$ e $dt$ são os parâmetros de discretização espacial e temporal respectivamente, $v_{\text{min}}$ e $v_{\text{max}}$ são as velocidades da onda P mínima e máxima no modelo, $f_{\text{max}}$ é a frequência máxima da fonte sísmica aplicada, e $\alpha$ e $\beta$ são as quantidades de amostras para representar um comprimento de onda no espaço e no tempo respectivamente. Os valores de $\alpha$ e $\beta$ variam de acordo com a ordem do operador de diferenças finitas \cite{bulcao2004modelagem}.    

A fonte sísmica pode assumir diferentes formas dependendo de seu formato ou equacionamento. Na modelagem sísmica, a fonte é aplicada como um pulso definido no tempo discreto, onde a cada passo de tempo uma amostra do sinal é injetada, em uma posição no espaço, na resolução da equação da onda. A segunda derivada da equação gaussiana \cite{ricker1953form} pode ser usada para gerar o pulso sísmico, sendo definida da seguinte maneira
\begin{equation}
	\begin{cases}
		f_p = f_{\text{max}} / (3\sqrt{\pi}) \\
		g(t) = (1 - 2 \pi (\pi f_p t)^2) \exp(-\pi (\pi f_p t)^2)
	\end{cases}
\end{equation}
\noindent sendo $f_p$ a frequência de pico, $g(t)$ a fonte sísmica e $t$ o eixo do tempo. 

Ao contrário de dados sísmicos reais, onde o planeta absorve por completo a energia da onda, em simulação computacional, para os casos onde a equação da onda não contempla fenômenos dissipativos, existem técnicas para prevenir reflexões indesejadas causadas pelas bordas do modelo finito empregado. A condição de bordas absortivas de \citeonline{cerjan1985nonreflecting} é uma abordagem clássica para resolver o problema de bordas na modelagem sísmica, porém outras formulações mais elaboradas podem ser utilizadas, uma possibilidade é aplicar a formulação descrita no trabalho de \citeonline{chern2019reflectionless}. A condição de borda utilizada neste trabalho é regida pela equação a seguir
\begin{equation}
	w(\mathbf{x}) = exp(- p(n_b - \mathbf{x})^2),	
\end{equation}      
\noindent onde $w(\mathbf{x})$ é a função amortecimento definida somente nos pontos da borda $\mathbf{x}$, $p$ é um fator de atenuação e $n_b$ é a quantidade de amostras na borda. \citeonline{bording1921finite} compara experimentos para determinar valores otimizados para $n_b$ e $p$, em contrapartida \citeonline{gao2017comparison} mostram que a técnica esponjosa absortiva foi superada pelas novas variantes utilizando modificações da equação da onda. 

Utilizando a equação da onda em simulações sísmicas, alguns tipos de onda são registradas, que para o caso acústico escalar somente a onda P, porém existem as frentes de onda refletidas e refratadas. Nas próximas seções, somente as frentes de onda transmitidas e refratadas são exploradas pois esse tipo de onda é a ferramenta principal deste trabalho. As ondas refratadas acontecem quando o afastamento entre a fonte e o receptor são suficientemente grandes, dependendo da configuração de camadas geológicas em subsuperfície. Essas ondas refratadas comumente são chamadas de primeiras chegadas, pois se propagam em camadas de maior velocidade e assim são registradas em um tempo menor que as demais frentes de onda.          

\subsection*{Equação \textit{eikonal}}

A equação eikonal pode ser definida como uma aproximação de altas frequências para a equação da onda em meios acústicos. Essa definição pode ser expandida para outras simplificações de meio, como por exemplo o meio elástico isotrópico \cite{cerveny2003seismic}. \citeonline{rawlinson2008seismic} mostram as etapas de simplificação partindo da equação \ref{wave_equation}, assumindo uma solução periódica envolvendo tempo e frequência angular, aplicando essa solução na equação da onda, simplificando os termos e aplicando o limite da frequência angular tendendo ao infinito. Assim, a equação eikonal em três dimensões segue no formato 
\begin{equation}
	\left[\dfrac{\partial T(\mathbf{x})}{\partial x}\right]^2 + \left[\dfrac{\partial T(\mathbf{x})}{\partial y}\right]^2 + \left[\dfrac{\partial T(\mathbf{x})}{\partial z}\right]^2 = \dfrac{1}{v^2(\mathbf{x})} 	
\end{equation} 
\noindent sendo $T(\mathbf{x})$ a função tempo de trânsito avaliada na posição $\mathbf{x} = (x,y,z)$ e $s = 1/v$ pode ser chamado de vagarosidade, o inverso da velocidade. 





\subsection*{Equação analítica para ondas refratadas}

livro do kearey \cite{kearey2002introduction}



\begin{equation}
	t_n = \dfrac{x}{v_n} + \displaystyle\sum_{i=1}^{n-1} 2z_i cos(\theta_{in})
\end{equation}

\begin{figure}[H]
	\centering
	\includegraphics[width = 11cm, height = 7cm]{Imgs/RevisaoBibliografica/refracted_analytic.png}
	\caption{}
	\label{fig:refracted_analytic}
\end{figure}



\subsection*{Método clássico}

\cite{vidale1988finite}
\cite{podvin1991finite}
\cite{afnimar2000finite}
\cite{tryggvason2006travel}

desenvolvimento com o algoritmo time3d








\subsection*{\textit{Fast Iterative Method}}

\cite{jeong2008fast}
\cite{sethian1999fast}

\cite{fu2011fast, fu2013fast}

\cite{dang2014fast, hong2016multi, hong2022mg}

\cite{cai2023improved}

desenvolvimento com a implementação 






\subsection*{\textit{Fast Sweeping Method}}

original fast sweeping method

\cite{zhao2005fast}

\cite{zhao2007parallel}

\cite{detrixhe2013parallel}

\cite{noble2014accurate}


\subsection*{Comparação numérica}


\begin{figure}[H]
	\centering
	\includegraphics[width = 11cm, height = 10cm]{Imgs/RevisaoBibliografica/modelGeometry.png}
	\caption{Modelo empregado no teste de precisão e performance. (a) Plano XY ilustrando a geometria de aquisição com o arranjo de receptores circulares possuindo somente um tiro central. Isócronas mapeando o comportamento dos tempos de trânsito são mostradas. (b) Perfil de velocidades delimitando a posição da interface. (c) e (d) são as projeções dos cortes em planos XZ e YZ em relação à posição da fonte.}
	\label{fig:configurationNumericalComparison}
\end{figure}


\begin{figure}[H]
	\centering
	\subfloat[]{\includegraphics[width=8cm,height=3.5cm]{Imgs/RevisaoBibliografica/precision_direct.png}\label{fig:rnca}}
	\subfloat[]{\includegraphics[width=8cm,height=3.5cm]{Imgs/RevisaoBibliografica/reciprocity.png}\label{fig:rncb}}
	
	\subfloat[]{\includegraphics[width=8cm,height=1.5cm]{Imgs/RevisaoBibliografica/error_pod_direct.png}\label{fig:rncc}}
	\subfloat[]{\includegraphics[width=8cm,height=1.5cm]{Imgs/RevisaoBibliografica/error_pod_reciprocity.png}\label{fig:rncd}}
	
	\subfloat[]{\includegraphics[width=8cm,height=1.5cm]{Imgs/RevisaoBibliografica/error_fim_direct.png}\label{fig:rnce}}
	\subfloat[]{\includegraphics[width=8cm,height=1.5cm]{Imgs/RevisaoBibliografica/error_fim_reciprocity.png}\label{fig:rncf}}
	
	\subfloat[]{\includegraphics[width=8cm,height=1.5cm]{Imgs/RevisaoBibliografica/error_fsm_direct.png}\label{fig:rncg}}
	\subfloat[]{\includegraphics[width=8cm,height=1.5cm]{Imgs/RevisaoBibliografica/error_fsm_reciprocity.png}\label{fig:rnch}}
	
	\caption{Comparação de precisão entre os métodos numéricos estudados. (a) Mapeamento de todas as chegadas para os métodos numéricos testados para os espaçamentos estudados. (b) Estudo de reciprocidade utilizando o espaçamento de 25 m. (c) Escala do erro para as chegadas em diferentes espaçamentos e (d) tempo direto e recíproco utilizando a formulação de \citeonline{podvin1991finite}. (e) Escala do erro e (f) estudo de reciprocidade para a formulação de \citeonline{jeong2008fast}. (g) Escala do erro e (h) estudo de reciprocidade para a formulação de \citeonline{noble2014accurate}.}
	\label{fig:resultsNumericalComparison}
\end{figure}







\section{Inversão tomográfica}

Tipos de tomografia (reflexão, difração, transmissão e refração)







trabalhos do GISIS

\cite{santos2012tomography}
\cite{bulhoes2021efeitos}



teoria de inversão

Função objetivo e linearização

Regularização 

\cite{seo2012nonlinear}
\cite{sain2023active}



least squares conjugate gradient

\cite{saad2003iterative}




\subsection*{Tomografia de refração}

Discretização do modelo

Raios ilustrativos



\subsection*{Obtenção do dado observado}



tipos de picking (manual, analítico, machine learning)

formulação utilizada

\cite{pan2019automatic} 
\cite{qin2021first} 

%========================================================================
% Modelo para elaboracao de textos academicos: TCC, dissertacoes e teses
% Elaborado pelo GISIS - Grupo de Imageamento Sismico e Inversao Sismica.
%========================================================================
\chapter{Metodologia}
\label{ch:metodologia}

Este capítulo dedica-se à minuciosa exposição de todas as informações referentes aos experimentos conduzidos para a avaliação meticulosa da precisão e do desempenho dos métodos de resolução da equação eikonal. Inicialmente, empreende-se uma análise comparativa que abrange tanto a precisão quanto o desempenho, abordando a solução numérica em modelos de propriedades simples e complexas. O escopo dessa comparação visa investigar a precisão em diferentes contextos, nomeadamente em um modelo homogêneo, em um modelo de duas camadas e em um modelo complexo amplamente empregado em análises sísmicas.

Os resultados concernentes ao desempenho são derivados de uma plataforma computacional singular, equipada com um processador Intel i7-12700 que dispõe de 20 núcleos, e uma placa gráfica NVIDIA GeForce RTX 3060 com uma capacidade de armazenamento de 12 GB. Destaca-se que os experimentos foram executados no ambiente do sistema operacional Windows 11, com a virtualização da distribuição Linux Ubuntu por meio do \textit{Windows Subsystem for Linux} (WSL). Ao detalhar cada faceta desses experimentos, busca-se oferecer uma compreensão abrangente e fundamentada acerca da eficácia e da aplicabilidade dos métodos de resolução da equação eikonal em levantamentos sísmicos, considerando diferentes cenários e características dos modelos utilizados.

%Todas as informações dos experimentos realizados para verificar a precisão e a performance dos métodos de resolução da equação eikonal estão descritas neste capítulo. Começando pela comparação de precisão e performance utilizando modelos de propriedades simples e complexos, para averiguar a precisão em um modelo homogêneo, em um modelo de duas camadas e em um modelo complexo amplamente utilizado em análises sísmicas. Os resultados de performance são gerados no mesmo computador tendo um processador Intel i7-12700 com 20 núcleos e uma placa gráfica NVIDIA GeForce RTX 3060 com 12 GB de armazenamento. O sistema operacional Windows 11 é utilizado com a virtualização da distribuição linux Ubuntu via WSL.  

\section{Aplicação em modelo homogêneo}

No decorrer deste estudo, optou-se por empregar um modelo homogêneo caracterizado por uma velocidade constante de 2000 m/s. Este modelo, que abrange dimensões espaciais de 200 $\times$ 200 $\times$ 200 metros, adota um parâmetro de discretização de 1 m, resultando em um total de 201 amostras em cada uma das direções $x$, $y$ e $z$. Destaca-se que a posição da fonte sísmica é centralizada no modelo, conferindo-lhe um caráter representativo e simétrico. A determinação do tempo analítico seguiu uma abordagem que envolveu o cálculo a partir da distância entre a fonte e cada ponto da malha, dividida pela velocidade do modelo. Dessa forma, cada ponto na malha foi associado a um tempo analítico correspondente, estabelecendo uma base para comparações posteriores. Cada método foi submetido a uma avaliação rigorosa em termos de eficiência computacional. Os tempos de execução foram registrados e comparados, destacando as nuances de desempenho entre os métodos. Este enfoque permite uma compreensão aprofundada não apenas da precisão, mas também da eficiência computacional de cada abordagem.

Dentre as diferentes formulações exploradas, merecem especial atenção aquelas propostas por \citeonline{podvin1991finite}, \citeonline{jeong2008fast} e \citeonline{noble2014accurate}. Além disso, como uma extensão valiosa deste estudo, o operador de maior precisão do \textit{Fast Iterative Mehod}, conforme delineado no trabalho de \citeonline{cai2023improved}, foi submetido ao teste com os demais formulações estudadas. Vale ressaltar que o código correspondente a esse operador foi gentilmente disponibilizado pelo autor, especificamente configurado para atender a um teste analítico no formato específico em questão. Na seção destinada à apresentação dos resultados, adota-se a prática de \citeonline{cai2023improved}, expondo apenas a média e o máximo erro. Essa abordagem se alinha não apenas com a acurácia ou performance, mas também com a consistência metodológica, contribuindo para uma compreensão mais clara e concisa dos resultados obtidos neste estudo. 

%Um modelo homogêneo foi aplicado com velocidade constante de 2000 m/s. O modelo tem 200 $\times$ 200 $\times$ 200 metros, com parâmetro de discretização de 1 m, totalizando 201 amostras em cada direção $x$, $y$ e $z$. A posição da fonte se encontra no centro do modelo. O tempo analítico é calculado a partir da distância da fonte a cada ponto da malha dividido pela velocidade do modelo. O erro é computado a partir da diferença entre o tempo numérico e o tempo analítico. As formulações de \citeonline{podvin1991finite}, \citeonline{jeong2008fast} e \citeonline{noble2014accurate} são analisadas. Como caso adicional, o operador mais preciso desenvolvido no trabalho de \citeonline{cai2023improved} é testado, pois o código foi fornecido pelo autor exatamente com um teste analítico nesse formado em questão. Na seção de resultados, somente a média e o máximo erro são mostrados assim como no trabalho de \citeonline{cai2023improved}. 

\section{Aplicação em modelo de refração}

No âmbito de um experimento com elevada relevância, cujos resultados foram submetidos à apreciação do Simpósio Brasileiro de Geofísica \cite{alves2022refraction}, será minuciosamente explorada uma comparação numérica entre as distintas formulações objeto de estudo. Este enfoque visa não apenas elucidar a precisão inerente ao cálculo de ondas refratadas, mas também analisar o comportamento desses métodos quando submetidos ao princípio da reciprocidade. O exemplo esquemático delineia uma configuração específica, na qual uma fonte é estrategicamente posicionada na superfície e no centro do modelo de velocidade. Concomitantemente, um arranjo circular de receptores, espaçados regularmente a intervalos de 50 metros, é distribuído a uma distância de 10 km da fonte. O modelo de velocidades adotado neste cenário particular apresenta apenas uma interface, caracterizada por propriedades de 1500 e 2000 m/s para cada camada. A representação visual dessa configuração experimental, evidenciada na Figura \ref{fig:configurationNumericalComparison}, inclui não apenas a disposição espacial da fonte e dos receptores, mas também um perfil detalhado do modelo de velocidades, bem como cortes do modelo com isócronas de tempo projetadas. No intuito de avaliar a precisão intrínseca dos métodos em questão, foram construídos três modelos de velocidade com distintos espaçamentos: 25, 50 e 100 metros. As dimensões e amostras destes modelos variam proporcionalmente à discretização adotada, sendo ($z$, $x$, $y$) = (12, 221, 221) amostras para o modelo de 100 m, (23, 441, 441) amostras para o modelo de 50 m e (45, 881, 881) amostras para o modelo de 25 m. Além dessa abordagem centrada na precisão das formulações, foi conduzido um estudo de reciprocidade, consistindo na inversão das posições da fonte e dos receptores. Essa análise em larga escala permitiu a observação de efeitos de inicialização em pontos situados fora da malha, contribuindo para uma compreensão mais abrangente do desempenho dos algoritmos estudados. Registrou-se, também, o tempo de execução do problema para 1257 tiros, proporcionando uma métrica temporal significativa para avaliar a eficiência dos métodos em questão. Este estudo compreensivo busca fornecer características valiosas sobre a aplicabilidade e desempenho dos métodos estudados em larga escala.


%Como parte de um experimento apresentado ao Simpósio Brasileiro de Geofísica \cite{alves2022refraction}, uma comparação numérica entre os métodos estudados será abordada, agora demonstrando a precisão no cálculo de ondas refratadas e o comportamento utilizando o princípio da reciprocidade. O exemplo esquemático aloca uma fonte na superfície do centro do modelo de velocidade e um arranjo circular de receptores espaçados regularmente de 50 metros com distância de 10 km da fonte. O modelo de velocidades empregado contém somente uma interface com propriedades de 1500 e 2000 m/s para cada camada. A figura \ref{fig:configurationNumericalComparison} mostra a configuração do experimento, o modelo de velocidades detalhado em forma de perfil e cortes do modelo com as isócronas de tempo projetadas. Para verificar a precisão dos métodos, foram construídos três modelos de velocidade com espaçamentos de 25, 50 e 100 metros. As amostras dos modelos construídos variam com a discretização do modelo sendo ($z$, $x$, $y$) = (12, 221, 221) amostras para o modelo de 100 m, (23, 441, 441) amostras para o modelo de 50 m e (45, 881, 881) amostras para o modelo de 25 m. Um estudo de reciprocidade, invertendo a posição da fonte com os receptores, foi efetuado para averiguar o funcionamento do algoritmo em larga escala. Assim, efeitos de inicialização em pontos fora da malha podem ser notados e o tempo de execução do problema para 1257 tiros pode ser registrado.

\begin{figure}[H]
	\centering
	\includegraphics[width = 11cm, height = 10.5cm]{Imgs/RevisaoBibliografica/modelGeometry.png}
	\caption{Modelo empregado no teste de precisão e performance. (a) Plano XY ilustrando a geometria de aquisição com o arranjo de receptores circulares possuindo somente um tiro central. Isócronas mapeando o comportamento dos tempos de trânsito são mostradas. (b) Perfil de velocidades delimitando a posição da interface. (c) e (d) são as projeções dos cortes em planos XZ e YZ em relação à posição da fonte.}
	\label{fig:configurationNumericalComparison}
\end{figure}

\section{Aplicação em modelo complexo}

Com o propósito de validar os algoritmos em um cenário mais próximo da realidade, as formulações estudadas foram aplicadas no modelo SEG/EAGE \textit{Overthrust}, cuja representação visual é proporcionada pela Figura \ref{fig:overthrust}. Este procedimento visa verificar o comportamento das frentes de onda e os tempos de execução diante de uma simulação caracterizada por elevados contrastes de velocidade. O modelo em questão possui dimensões de ($z$, $x$, $y$) = (4.5, 20, 20) km, adotando um parâmetro de discretização de 25 m, o que resulta em um total de (181, 801, 801) amostras. Na configuração específica desse experimento, foi empregado um esquema de geometria circular contendo três círculos na superfície do modelo, distanciados a 5500, 7500 e 9500 metros do centro, onde a posição da fonte está situada. O espaçamento entre os receptores foi estabelecido em 25 m, totalizando 5662 estações, buscando registrar detalhes precisos dos tempos gerados pelas formulações responsáveis por resolver a equação eikonal. Como resultado desse experimento, janelas de aproximação nos dados evidenciam os detalhes que o modelo de alto contraste projeta nos dados calculados. Outro aspecto analisado refere-se aos tempos de execução das formulações propostas por \citeonline{podvin1991finite}, \citeonline{jeong2008fast} e \citeonline{noble2014accurate}, os quais foram mensurados após o cálculo dos tempos das ondas de primeira chegada. 

%A fim de validar os algoritmos em um modelo realístico, as formulações estudadas foram aplicadas no modelo SEG/EAGE \textit{Overthrust}, ilustrado na Figura \ref{fig:overthrust}, na intenção de verificar o comportamento das frentes de onda e os tempos de execução perante uma simulação com altos contrastes de velocidade. O modelo possui dimensões de  ($z$, $x$, $y$) = (4.5, 20, 20) km e parâmetro de discretização de 25 m, totalizando (181, 801, 801) amostras. O esquema de geometria circular foi utilizado possuindo três círculos na superfície do modelo com afastamentos 5500, 7500 e 9500 metros do centro do modelo, onde a posição da fonte se encontra. O espaçamento entre os receptores é de 25 m, totalizando 5662 estações, para registrar o máximo de detalhes possível dos tempos gerados pelas formulações que resolvem a equação eikonal. Como o resultado desse experimento, as primeiras chegadas são apresentadas de forma geral e janelas de aproximação no dado mostram os detalhes que o modelo de alto contraste projeta nos dados calculados. Outro resultado são os tempos de execução das formulações de \citeonline{podvin1991finite}, \citeonline{jeong2008fast} e \citeonline{noble2014accurate} após o cálculo do tempo das ondas de primeira chegada. 

\begin{figure}[H]
	\centering
	\includegraphics[width = 7.5cm, height = 7.9cm]{Imgs/Metodologia/overthrust.png}
	\caption{Modelo SEG/EAGE \textit{Overthrust} para aplicação dos métodos em altos contrastes de velocidade. A geometria circular é aplicada para verificar o comportamento azimutal dos tempos de trânsito. Em preto são os receptores espaçados em 25 metros totalizando 5662 estações. Uma fonte é a plicada no centro do modelo.}
	\label{fig:overthrust}
\end{figure}








%========================================================================
% Modelo para elaboracao de textos academicos: TCC, dissertacoes e teses
% Elaborado pelo GISIS - Grupo de Imageamento Sismico e Inversao Sismica.
%========================================================================
\chapter{Resultados e Discussões}
\label{ch:resultados}

Neste capítulo, são apresentados os resultados da comparação entre os métodos de \citeonline{podvin1991finite}, \citeonline{jeong2008fast} e \citeonline{noble2014accurate} em termos de precisão no cálculo dos tempos de trânsito e desempenho computacional. Assim que apresentados, ambos recebem suas respectivas discussões individualmente.

A aplicação com modelo homogêneo replica o experimento realizado no trabalho de \citeonline{cai2023improved}, onde a inovação em precisão foi desenvolvida para o \textit{Fast Iterative Method}. A aplicação com modelo de duas camadas replica o experimento contido em \citeonline{alves2022refraction}. E a aplicação em modelo complexo mostra como os tempos de trânsito se comportam na presença de altos contrastes de velocidade e qual o tempo de execução para cada método no modelo mais realístico.    

\section{Aplicação em modelo homogêneo}

A Tabela \ref{table_homog} apresenta os tempos de execução, o erro médio e o erro máximo para cada formulação. Os resultados gerados utilizando a implementação de \citeonline{cai2023improved} foram executados no mesmo ambiente, segundo as configurações apresentadas no capítulo de Metodologia, devido a disponibilização do código publicamente.  

\begin{table}[H]
	\caption{Tempo de execução, erro médio e máximo em relação a equação analítica para meio homogêneo. Aplicação dos métodos utilizados relacionando com os resultados do experimento de \citeonline{cai2023improved}.}
	\begin{tabular}{r|ccc}
		\multicolumn{1}{c|}{} & Tempo {[}s{]} & Erro médio {[}s{]} & Erro máximo {[}s{]} \\ \hline
		\citeonline{podvin1991finite} & 0,9661        & 0,000273           & 0,000762            \\ \hline
		\citeonline{jeong2008fast}    & 0,6082        & 0,000837           & 0,001323            \\ \hline
		\citeonline{noble2014accurate}& 0,8085        & 0,000037           & 0,000069            \\ \hline
		\citeonline{cai2023improved}  & 1,9312        & 0,000196           & 0,000282           
	\end{tabular}
	\label{table_homog}
\end{table}

No teste do modelo homogêneo (Tabela \ref{table_homog}) nota-se a superioridade em performance do método de \citeonline{jeong2008fast}, porém os erros registrados são os mais altos em relação aos outros métodos testados. O método de \citeonline{cai2023improved} consegue reduzir o erro do \textit{Fast Iterative Method}, contudo, além de não ser o mais preciso entre os métodos analisados, a nova formulação demostrou performance inferior em relação às demais estudadas. O método que demonstrou mais precisão foi a nova implementação do \textit{Fast Sweeping Method} utilizando a estratégia de paralelização de \citeonline{detrixhe2013parallel} e os operadores desenvolvidos no trabalho de \citeonline{noble2014accurate}. A eficiência computacional do \textit{Fast Sweeping Method} não se destacou como a mais performática, porém se demonstra promissora ao considerar o conjunto entre precisão e performance.  

\section{Aplicação em modelo de refração}

As Figuras \ref{fig:general_refraction_study}, \ref{fig:precision_refraction_study} e \ref{fig:reciprocity_refraction_study} exibem todo o estudo de precisão com os tempos calculados da fonte para os receptores (tempos diretos) e dos receptores para a fonte (tempos recíprocos). As cores indicam os métodos, sendo o azul para a formulação de \citeonline{podvin1991finite}, o amarelo para a formulação de \citeonline{jeong2008fast} e o verde para a formulação de \citeonline{noble2014accurate}. Os estilos de linha representam o parâmetro de discretização, então, as linhas sólidas representam o modelo de 25 m, as linhas tracejadas, o de 50 m, e as linhas com ponto e traço, o de 100 m. 

\begin{table}[H]
	\caption{Tempo de execução para cada discretização e reciprocidade para o modelo de 25 m.}
	\begin{tabular}{r|cccc}
		& 100 m    & 50 m     & 25 m     & Reciprocidade \\ \hline
		\citeonline{podvin1991finite}   & 0,0676 s & 0,3141 s & 3,5324 s & 7896,2 s        \\ \hline
		\citeonline{jeong2008fast} & 0,0491 s & 0,0987 s & 0,6261 s & 921,8 s              \\ \hline
		\citeonline{noble2014accurate} & 0,1042 s & 0,2286 s & 0,8284 s & 1382,5 s          \\
	\end{tabular}
	\label{table_refModel}
\end{table}

A Tabela \ref{table_refModel} mostra, para cada método testado, um teste de escalabilidade, onde o domínio do problema cresce para verificar o desempenho computacional dos métodos numéricos. A análise de reciprocidade foi crucial para verificar extravasamentos de memória quando aplicadas múltiplas posições de tiro. A validação em aplicações de imageamento sísmico também pode ser observada a partir do estudo de reciprocidade, pois o tempo de execução importa na geração de resultados. 

A Figura \ref{fig:general_refraction_study} mostra a configuração de todos os métodos para o tiro central, sendo o tempo analítico a linha sólida de cor preta. A Figura \ref{fig:precision_refraction_study} mostra a ordem do erro para cada método e a Figura \ref{fig:reciprocity_refraction_study} mostra as diferenças entre o tempo direto e recíproco para cada método utilizando o modelo de duas camadas com 25 m de discretização.  

\begin{figure}[H]
	\centering
	\subfloat[]{\includegraphics[width=16cm,height=6cm]{Imgs/RevisaoBibliografica/precision_direct.png}\label{fig:rnca}}\newline
	\subfloat[]{\includegraphics[width=16cm,height=6cm]{Imgs/RevisaoBibliografica/reciprocity.png}\label{fig:rncb}} 
	\caption{Panorama geral do comportamento dos tempos de trânsito para todos os métodos estudados. (a) exibe a variação para cada discretização do modelo de velocidade e (b) mostra somente o estudo de reciprocidade aplicado no modelo de 25 m.}
	\label{fig:general_refraction_study}	
\end{figure}

A solução clássica demonstrou desempenho computacional razoável para problemas pequenos, porém quando a dimensão do problema aumenta os tempos de execução não se adequam aos demais métodos. Sendo assim a formulação clássica diminui consideravelmente sua performance, no modelo com 25 m de discretização, enquanto as demais formulações mantêm seus patamares abaixo de um segundo de tempo de execução. O \textit{Fast Iterative Method} demonstrou o melhor desempenho computacional em todas as análises. Para problemas onde a precisão do cálculo dos tempos de trânsito é requerida, o \textit{Fast Iterative Method} não é tão recomendado, porém para problemas de \textit{Path Fiding} ou \textit{Shape from Shading} o método de \citeonline{jeong2008fast} é extremamente recomendável. Com a implementação em paralelo utilizando a técnica de \citeonline{detrixhe2013parallel} a variação do \textit{Fast Sweeping Method} com operadores precisos de \citeonline{noble2014accurate} demonstrou uma performance comparável ao método de \citeonline{jeong2008fast} principalmente quando aplicado no modelo de menor discretização.  

\begin{figure}[H]
	\centering
	\subfloat[]{\includegraphics[width=16cm,height=3cm]{Imgs/RevisaoBibliografica/error_pod_direct.png}\label{fig:rncc}}\newline
	\subfloat[]{\includegraphics[width=16cm,height=3cm]{Imgs/RevisaoBibliografica/error_fim_direct.png}\label{fig:rnce}}\newline
	\subfloat[]{\includegraphics[width=16cm,height=3cm]{Imgs/RevisaoBibliografica/error_fsm_direct.png}\label{fig:rncg}}\newline
		
	\caption{Diferença entre os tempos analítico $t_a$ e numérico $t_n$ por método em relação aos parâmetros de discretização do modelo. (a) o método clássico de \citeonline{podvin1991finite}, (b) o FIM e (c) o FSM utilizando discretizações de 100, 50 e 25 m.}
	\label{fig:precision_refraction_study}	
\end{figure}

A Figura \ref{fig:precision_refraction_study} mostra que a diferença entre a solução analítica e a solução numérica aumentam de acordo com a esparsidade da malha, ou seja, os erros são proporcionais ao refinamento do modelo como consequência das limitações do método das diferenças finitas. A solução para melhorar a precisão é aumentar o operador de diferenças finitas para utilizar mais pontos vizinhos no cálculo dos tempos de trânsito \cite{noble2014accurate,cai2023improved}. Outra característica a se destacar na Figura \ref{fig:precision_refraction_study} são os formatos dos erros sendo periódicos de acordo com a geometria circular, causados pela formulação em coordenadas cartesianas \cite{white2020pykonal}. 

A relação de precisão entre os métodos é bem distante, sendo a formulação clássica com erros intermediários, o \textit{Fast Sweeping Method} com erros pequenos e o \textit{Fast Iterative Method} com maiores erros (Figura \ref{fig:precision_refraction_study}). Sendo assim, para problemas geofísicos, o \textit{Fast Sweeping Method} se destaca com sua performance aceitável para a precisão que proporciona, sendo o mais recomendável neste trabalho.       

\begin{figure}[H]
	\centering
	\subfloat[]{\includegraphics[width=16cm,height=3cm]{Imgs/RevisaoBibliografica/error_pod_reciprocity.png}\label{fig:rncd}}\newline
	\subfloat[]{\includegraphics[width=16cm,height=3cm]{Imgs/RevisaoBibliografica/error_fim_reciprocity.png}\label{fig:rncf}}\newline
	\subfloat[]{\includegraphics[width=16cm,height=3cm]{Imgs/RevisaoBibliografica/error_fsm_reciprocity.png}\label{fig:rnch}}
	
	\caption{Sobreposição entre os erros dos tempos de trânsito diretos, da fonte para os receptores, e recíprocos, dos receptores para a fonte, em relação à equação analítica. Diferença analítico $t_a$ e numérico $t_n$ em relação aos métodos (a) clássico, (b) FIM e (c) FSM somente para a discretização de 25 m entre os pontos da malha.}  
	\label{fig:reciprocity_refraction_study}
\end{figure}

A superioridade de precisão do \textit{Fast Sweeping Method} pode ser identificada também na Figura \ref{fig:reciprocity_refraction_study}, porém no estudo de reciprocidade a inicialização causou instabilidade na modelagem, como mostrado na Figura \ref{fig:reciprocity_refraction_study}c. Para iniciar o cálculo dos tempos de trânsito, todo o volume é inicializado com um valor tendendo ao infinito e somente os pontos vizinhos da fonte são inicializados com o tempo analítico para velocidade constante (equação \ref{analyticalT}). O algoritmo original de \citeonline{noble2014accurate} inicializa analiticamente somente os pontos do primeiro octante, porém o estudo de reciprocidade mostrou que a inicialização deve ser feita para os pontos vizinhos em todas as direções, ou seja,  nos oito octantes vizinhos da fonte como mostrado na Figura \ref{fig:voxel_full}. Após as correções, uma nova modelagem utilizando reciprocidade foi performada validando ainda mais a precisão do \textit{Fast Sweeping Method}, como mostrado, em linha tracejada verde, na Figura \ref{fig:reciprocity_refraction_study}c.


\section{Aplicação em modelo complexo}

As Figuras \ref{fig:overthrust_inner_circle}, \ref{fig:overthrust_mid_circle} e \ref{fig:overthrust_outer_circle} mostram os resultados gerados a partir do esquema de modelagem com geometria circular aplicado no modelo SEG/EAGE \textit{Overthrust}. Novamente as cores da figura indicam cada método, sendo a cor azul para a formulação de \citeonline{podvin1991finite}, laranja para \citeonline{jeong2008fast} e verde para a formulação de \citeonline{noble2014accurate}. As figuras são ampliações do sismograma de primeira chegada com uma seleção de estações que evidencia a disparidade entre os tempos de trânsito. Como a equação eikonal é uma aproximação da equação da onda para altas frequências, espera-se que os detalhes do modelo sejam identificados no dado, então, as janelas amplificadoras facilitam a identificação desses aspectos.    

\begin{figure}[H]
	\centering
	\includegraphics[height=8cm,width=13cm]{Imgs/Resultados/complex_w1.png}		
	\caption{Tempos de trânsito capturados do circulo menor de 5,5 km. As cores representam cada solução numérica, sendo azul para o método clássico, laranja para o FIM e verde para o FSM. O eixo horizontal representa o índice dos receptores entre as 5662 estações e o eixo vertical é o tempo de percurso da frente de onda.}
	\label{fig:overthrust_inner_circle}
\end{figure}

O atraso intrínseco das formulações de \citeonline{podvin1991finite} e \citeonline{jeong2008fast} pode ser notado no modelo complexo assim como foi notado utilizando o modelo simples com duas camadas. A formulação de \citeonline{noble2014accurate} se mostra superior em relação aos atrasos no tempo de trânsito por consequência dos tempos maiores observados nos demais métodos.

\begin{figure}[H]
	\centering
	\includegraphics[height=8cm,width=13cm]{Imgs/Resultados/complex_w2.png}	
	\caption{Tempos de trânsito capturados do circulo intermediário de 7,5 km. As cores representam cada solução numérica, sendo azul para o método clássico, laranja para o FIM e verde para o FSM. O eixo horizontal representa o índice dos receptores entre as 5662 estações e o eixo vertical é o tempo de percurso da frente de onda.}
	\label{fig:overthrust_mid_circle}
\end{figure}

\begin{figure}[H]
	\centering
	\includegraphics[height=8cm,width=13cm]{Imgs/Resultados/complex_w3.png} \newline 
	
	\caption{Tempos de trânsito capturados do círculo maior de 9,5 km. As cores representam cada solução numérica, sendo azul para o método clássico, laranja para o FIM e verde para o FSM. O eixo horizontal representa o índice dos receptores entre as 5662 estações e o eixo vertical é o tempo de percurso da frente de onda.}
	\label{fig:overthrust_outer_circle}
\end{figure}

A Tabela \ref{table_overthrust} apresenta os tempos de execução para cada método utilizando o modelo complexo em geometria circular aplicando somente um tiro no centro do modelo. A superioridade de performance é atribuída ao método de \citeonline{jeong2008fast}, porém essa metodologia possui atrasos severos em relação aos tempos de trânsito calculados comparado à equação analítica e aos demais métodos testados.

\begin{table}[H]
	\caption{Tempo de execução para a aplicação dos métodos no modelo complexo.}
	\begin{tabular}{r|c}
		& Tempo de execução \\ \hline
		\citeonline{podvin1991finite} & 11,7128 s  \\ \hline
		\citeonline{jeong2008fast} & 2,5595 s      \\ \hline
		\citeonline{noble2014accurate} & 3,4746 s         
	\end{tabular}
	\label{table_overthrust}
\end{table}

A aplicação dos métodos no modelo complexo foi realizada com sucesso, então, os tempos de trânsito (Figuras \ref{fig:overthrust_inner_circle}, \ref{fig:overthrust_mid_circle} e \ref{fig:overthrust_outer_circle}) e de execução (Tabela \ref{table_overthrust}) foram registrados. Novamente, o método clássico demonstrou tempo de execução acima dos demais para problemas grandes. O método de \citeonline{jeong2008fast} demonstra eficiência computacional e o \textit{Fast Sweeping Method} performa 26$\%$ abaixo do método mais performático. Considerando o ganho de precisão, a diferença de performance pode ser desconsiderada pois \citeonline{cai2023improved} mostram a utilização do método original de \citeonline{jeong2008fast} na migração Kirchhoff em profundidade, onde os erros de posicionamento dos refletores são amplamente perceptíveis. Os tempos de trânsito ilustrados nas Figuras \ref{fig:overthrust_inner_circle}, \ref{fig:overthrust_mid_circle} e \ref{fig:overthrust_outer_circle} mostram as discrepâncias entre os métodos aplicados ao modelo SEG/EAGE \textit{Overthrust}, sendo que os atrasos percebidos na Figura \ref{fig:precision_refraction_study} se repetem, porém no modelo complexo. Se mostra visível nas Figuras \ref{fig:overthrust_inner_circle}, \ref{fig:overthrust_mid_circle} e \ref{fig:overthrust_outer_circle} os detalhes que o modelo impõe aos tempos de trânsito, nas formulações de \citeonline{podvin1991finite} e \citeonline{noble2014accurate}, e a falta de definição para a formulação menos precisa de \citeonline{jeong2008fast}.  






%%========================================================================
% Modelo para elaboracao de textos academicos: TCC, dissertacoes e teses
% Elaborado pelo GISIS - Grupo de Imageamento Sismico e Inversao Sismica.
%========================================================================
\chapter{Discussões}
\label{ch:discussoes}

% Focar nos impactos da imprecisão no cálculo dos tempos na inversão
% Analisar as diferenças e ruídos de inversão para cada método
% Analisar performance em relação aos resultados
 
\begin{figure}[H]
	\centering
	\includegraphics[width=12cm,height=7cm]{Imgs/Discussoes/discuss_geometry.png}
	\caption{Panorama de discussão envolvendo o dado e os modelos. Em relação aos dados, que estão ordenados no domínio do receptor, o ponto vermelho de tonalidade marcante será o receptor de referência. Os pontos azuis em destaque são as linhas de tiros utilizadas apara a análise. Em relação ao modelo, cinco pontos foram destacados nas mesmas posições das gaussianas utilizadas para gerar o modelo de referência.}
	\label{fig:discuss_geometry}	
\end{figure}

\begin{figure}[H]
	\centering
	\subfloat[]{\includegraphics[width=8cm,height=8cm]{Imgs/Discussoes/xz_zoom_out.png}}
	\subfloat[]{\includegraphics[width=8cm,height=8cm]{Imgs/Discussoes/yz_zoom_out.png}}
	
	\caption{Comparação de dados incluindo o dado observado, linha preta sólida, dado inicial, linhas coloridas com traço e ponto, dado gerado a partir do primeiro processo de inversão, linhas coloridas sólidas, e o dado gerado a partir do segundo processo de inversão, linhas coloridas tracejadas. Em relação aos métodos estudados, a cor azul se refere à formulação de \citeonline{podvin1991finite}, a cor amarela \citeonline{jeong2008fast} e a cor verde \citeonline{jeong2008fast}.}
	\label{fig:zoom_out}
\end{figure}

\begin{figure}[H]
	\centering
	\subfloat[]{\includegraphics[width=8cm,height=8cm]{Imgs/Discussoes/xz_zoom_in.png}}
	\subfloat[]{\includegraphics[width=8cm,height=8cm]{Imgs/Discussoes/yz_zoom_in.png}}
	
	\caption{Aproximação para melhor analise dos dados. Dado observado se apresenta como a linha preta sólida, o dado inicial, linhas coloridas com traço e ponto, dado gerado com modelo da inversão esparsa, linhas coloridas sólidas, e o dado gerado com modelo da inversão refinada, linhas coloridas tracejadas. Em relação aos métodos estudados, a cor azul se refere à formulação de \citeonline{podvin1991finite}, a cor amarela \citeonline{jeong2008fast} e a cor verde \citeonline{jeong2008fast}.}
	\label{fig:zoom_in}
\end{figure}


\begin{figure}[H]
	\centering
	\subfloat[]{\includegraphics[width=3.2cm,height=6cm]{Imgs/Discussoes/upper_left_corner.png}}
	\subfloat[]{\includegraphics[width=3.2cm,height=6cm]{Imgs/Discussoes/lower_left_corner.png}}
	\subfloat[]{\includegraphics[width=3.2cm,height=6cm]{Imgs/Discussoes/center.png}}
	\subfloat[]{\includegraphics[width=3.2cm,height=6cm]{Imgs/Discussoes/upper_right_corner.png}}
	\subfloat[]{\includegraphics[width=3.2cm,height=6cm]{Imgs/Discussoes/lower_right_corner.png}}
	
	\caption{Modelos finais em forma de traço em cada posição central das gaussianas utilizadas. (a) Modelos projetados no ponto superior esquerdo. (b) Projeção no ponto inferior esquerdo. (c) Projeção central. (d) Projeção no ponto superior direito. (e) Projeção no ponto inferior direito. Em relação às cores, o modelo de referência se apresenta como o traço preto sólido e o modelo inicial como o traço vermelho sólido. Traços coloridos sólidos se referem aos modelos recuperados com malha esparsa e traços coloridos tracejados são modelos recuperados com malha refinada. Os métodos estudados recebem as cores: azul para \citeonline{podvin1991finite}, amarela para \citeonline{jeong2008fast} e verde para \citeonline{jeong2008fast}.}
	\label{fig:}
\end{figure}




%========================================================================
% Modelo para elaboracao de textos academicos: TCC, dissertacoes e teses
% Elaborado pelo GISIS - Grupo de Imageamento Sismico e Inversao Sismica.
%========================================================================
\chapter{Conclusão}
\label{ch:conclusao}

Neste trabalho, são analisados e desenvolvidos três métodos numéricos: o método clássico, o \textit{Fast Iterative Method} original e a versão paralelizada do \textit{Fast Sweeping Method} com operadores modificados. Esses métodos são aplicados para resolver a equação eikonal, com o objetivo de avaliar tanto a eficiência computacional quanto a precisão das soluções em comparação com soluções analíticas. Para realizar essa avaliação, foram considerados três cenários distintos: a propagação em meio homogêneo, a aplicação em um modelo simples de duas camadas e a propagação em um modelo geológico complexo com significativos contrastes de velocidade.

Os resultados obtidos na investigação do modelo homogêneo apontam para uma constatação relevante: apesar das melhorias implementadas no \textit{Fast Iterative Method} visando aprimorar a precisão, essa formulação não conseguiu superar a abordagem mais precisa desenvolvida neste estudo. Nesse cenário, o método que se destacou pela melhor acurácia foi o \textit{Fast Sweeping Method}. Os valores médio e máximo de erro apresentados na tabela correspondente reforçam essa superioridade, indicando que, ao confrontar as soluções numéricas com as expectativas analíticas, o \textit{Fast Sweeping Method} alcançou desempenho mais confiável e consistente. Essa constatação respalda a escolha desse método como uma opção mais eficaz e precisa para a resolução da equação eikonal em contextos homogêneos.

No que concerne ao estudo de onda refratada, considerando o modelo de duas camadas e variando a quantidade de amostras através do parâmetro de discretização, avaliamos não apenas a precisão, mas também o custo computacional associado a cada método, tanto para uma propagação simples quanto para múltiplas propagações no estudo de reciprocidade. Os resultados indicam que, à medida que refinamos o modelo de velocidade, os erros tendem a diminuir, mas é notável que a formulação mais eficiente computacionalmente apresenta maiores magnitudes de erro. Surpreendentemente, a formulação mais precisa, o \textit{Fast Sweeping Method}, demonstra desempenho comparável ao método mais eficiente, o \textit{Fast Iterative Method}, embora mantenha a melhor precisão. Os estudos de reciprocidade revelaram uma falha na inicialização do método mais preciso, porém esse erro foi corrigido ao inicializar todos os pontos vizinhos à fonte. Em uma escala mais ampla, o \textit{Fast Iterative Method} se destacou em termos de desempenho, mas, paradoxalmente, apresentou os maiores erros em relação à precisão nesse contexto específico.

Os métodos numéricos não enfrentaram dificuldades durante a aplicação no modelo de geologia complexa. Os resultados revelam um atraso intrínseco nas formulações clássica e no \textit{Fast Iterative Method}. Essa defasagem acaba resultando em um modelo com velocidades subestimadas, prejudicando etapas cruciais de imageamento, como a tomografia e a migração, que utilizam os tempos de trânsito gerados pela equação eikonal. No que diz respeito ao desempenho computacional dos métodos no modelo realista, observa-se que a formulação clássica alcança os níveis mais baixos de performance, ao passo que as formulações mais recentes se destacam. Apesar da formulação mais precisa não ser a mais eficiente computacionalmente, os valores de precisão que proporciona são significativamente superiores em relação ao método mais eficiente, apresentando uma diferença de apenas 26\% entre as formulações testadas.

\section{Trabalhos futuros}

Para futuras investigações, a partir da modelagem mais precisa realizada neste trabalho, é possível explorar a implementação de técnicas avançadas de imageamento e inversão sísmica, especialmente voltadas para a resolução de problemas em alta definição. Uma abordagem sugerida é o desenvolvimento de algoritmos de tomografia que empreguem operadores do estado adjunto. Nesse contexto, a utilização da equação eikonal se torna essencial, pois uma formulação precisa pode contribuir significativamente para a recuperação de valores de velocidade mais fidedignos. Além disso, outra direção promissora para pesquisas futuras é a aplicação da migração Kirchhoff em profundidade. Ao empregar uma formulação eikonal mais precisa, os eventos sísmicos podem ser representados de maneira mais acurada, minimizando potenciais erros na conversão entre tempo e profundidade. Essas abordagens não apenas aprimorariam a compreensão do subsolo, mas também teriam aplicações práticas em diversas áreas, incluindo exploração de recursos naturais e estudos geotécnicos.


% ----------------------------------------------------------
% ELEMENTOS PÓS-TEXTUAIS
% ----------------------------------------------------------
\postextual
% ----------------------------------------------------------

% ----------------------------------------------------------
% Referências bibliográficas
% ----------------------------------------------------------
\bibliography{referencias}


% ----------------------------------------------------------
% Glossário
% ----------------------------------------------------------
%
% Consulte o manual da classe abntex2 para orientações sobre o glossário.
%
%\glossary

% ----------------------------------------------------------
% Apêndices
% ----------------------------------------------------------
% ---
% Inicia os apêndices
% ---
%\begin{apendicesenv}
%
%    % Imprime uma página indicando o início dos apêndices
%    \partapendices
%
%    %========================================================================
% Modelo para elaboracao de textos academicos: TCC, dissertacoes e teses
% Elaborado pelo GISIS - Grupo de Imageamento Sismico e Inversao Sismica.
%========================================================================
% ----------------------------------------------------------
\chapter{Informações sobre apêndices}
% ----------------------------------------------------------
\label{append:info}

O apêndice é um elemento pós-textual que faz parte dos trabalhos de pesquisa. É utilizado para complementar ou comprovar a argumentação do texto. Pode ser um texto ou documento, elaborado pelo próprio autor.

Os apêndices devem aparecer depois das referências.
%
%\end{apendicesenv}
% ----------------------------------------------------------
% Anexos
% ----------------------------------------------------------

% ---
% Inicia os anexos
% ---
\begin{anexosenv}
    \partanexos
    %========================================================================
% Modelo para elaboracao de textos academicos: TCC, dissertacoes e teses
% Elaborado pelo GISIS - Grupo de Imageamento Sismico e Inversao Sismica.
%========================================================================
% ----------------------------------------------------------
\chapter{Resumo expandido 2022}
% ----------------------------------------------------------

\begin{center}
	\textbf{3D refraction travel times accuracy study in high contrasted media}
\end{center}

\begin{center}
	 IX Simpósio Brasileiro de Geofísica, Resumo expandido, 2022. Curitiba, PR, Brasil.
\end{center}

\begin{center}
	\textbf{Alves, P. H. B.}; Capuzzo, F. V.; Cetale, M. \& Santos, L. A.    
\end{center}

\includepdf[pages=-]{eikonalComparison_SimBGf_Curitiba.pdf}

    %========================================================================
% Modelo para elaboracao de textos academicos: TCC, dissertacoes e teses
% Elaborado pelo GISIS - Grupo de Imageamento Sismico e Inversao Sismica.
%========================================================================
% ----------------------------------------------------------
\chapter{Resumo expandido 2023}
% ----------------------------------------------------------

\begin{center}
	\textbf{First arrival tomography application in low illumination time-lapse reservoir monitoring}
\end{center}

\begin{center}
	18th International Congress of the Brazilian Geophysical Society \& EXPOGEF. 
	Rio de Janeiro, Brazil. 2023. SBGf. 
\end{center}

\begin{center}
	\textbf{Alves, P. H. B.}; Moreira, R. M.; Cetale, M. \& Lopez, J.    
\end{center}

\includepdf[pages=-]{CISBGf_2023_tomography.pdf}
    %========================================================================
% Modelo para elaboracao de textos academicos: TCC, dissertacoes e teses
% Elaborado pelo GISIS - Grupo de Imageamento Sismico e Inversao Sismica.
%========================================================================
% ----------------------------------------------------------
\chapter{Submissão de artigo 2023}
% ----------------------------------------------------------
\includepdf[pages=-]{BrJG_Paulo_eikonal_paper}


\end{anexosenv}

%---------------------------------------------------------------------
% INDICE REMISSIVO
%---------------------------------------------------------------------
%\phantompart
%\printindex
%---------------------------------------------------------------------

\end{document}
