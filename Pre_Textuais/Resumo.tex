%========================================================================
% Modelo para elaboracao de textos academicos: TCC, dissertacoes e teses
% Elaborado pelo GISIS - Grupo de Imageamento Sismico e Inversao Sismica.
%========================================================================
\setlength{\absparsep}{18pt} % ajusta o espaçamento dos parágrafos do resumo
\begin{resumo}
    \fonteResumo

	\begin{flushleft}
	\MakeUppercase{\imprimirSobrenome}, \imprimirPrimeirosNomes. \textbf{\imprimirtitulo}. \imprimirtipotrabalho, \imprimirinstituicao. \imprimirlocal,  p. \pageref{LastPage}. \imprimirAnoDeDefesa.
	\end{flushleft}
	
	Este estudo avalia a precisão e a eficiência computacional de três métodos distintos para resolver a equação eikonal, a qual representa uma aproximação de alta frequência da equação da onda em meios acústicos caracterizados por densidade constante. Os métodos numéricos analisados consistem, primeiramente, em uma formulação clássica bem estabelecida na literatura, em segundo lugar, em um método com desbravamento limitado no âmbito das geociências, porém com desempenho computacional significativo, e, por fim, a inovação desenvolvida neste estudo, a união entre duas formulações para gerar uma abordagem computacional precisa e eficiente. A implementação dessas formulações, utilizando programação paralela em placas gráficas, demandou uma validação rigorosa da precisão das soluções, a fim de garantir a viabilidade dos métodos numéricos em contextos tridimensionais complexos. Para verificar a precisão das soluções, foram adotadas três estratégias, incluindo a aplicação dos métodos em um modelo homogêneo, no qual uma formulação recente foi testada de maneira restrita, a avaliação em um modelo de duas camadas com variações no tamanho, permitindo a análise da escalabilidade dos algoritmos, e, por fim, a análise em um modelo realista que incorpora geologia complexa, fornecendo resultados detalhados sobre a eficiência computacional. Entre todas as formulações examinadas, o \textit{Fast Sweeping Method} se destacou pela sua notável precisão no cálculo dos tempos de trânsito, impulsionado pela utilização de operadores precisos paralelizados em placas gráficas.

    \textbf{Palavras-chave}: equação eikonal; sísmica de refração; comparação analítica-numérica; computação de alto desempenho.
\end{resumo}
