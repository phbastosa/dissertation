%========================================================================
% Modelo para elaboracao de textos academicos: TCC, dissertacoes e teses
% Elaborado pelo GISIS - Grupo de Imageamento Sismico e Inversao Sismica.
%========================================================================
\setlength{\absparsep}{18pt} % ajusta o espaçamento dos parágrafos do resumo
\begin{resumo}
    \fonteResumo

\begin{flushleft}
\MakeUppercase{\imprimirSobrenome}, \imprimirPrimeirosNomes. \textbf{\imprimirtitulo}. \imprimirtipotrabalho, \imprimirinstituicao. \imprimirlocal,  p. \pageref{LastPage}. \imprimirAnoDeDefesa.
\end{flushleft}

	Com o avanço computacional, a performance de soluções numéricas geram maiores discussões à precisão quando se trata de problemas envolvendo a equação eikonal. Neste trabalho, além da performance, três métodos são validados em relação à precisão aplicados em diferentes cenários de modelagem sísmica utilizando uma solução analítica para ondas refratadas. Em conjunto com a análise de precisão, um experimento de tomografia sísmica é realizado para averiguar as diferenças na reconstrução do modelo de velocidades com base nos métodos numéricos estudados. Uma solução clássica sedimentada em geociências, um novo método pouco explorado e uma solução que promete ser devidamente acurada são analisadas. Dois esquemas de inversão são propostos para identificar os efeitos da imprecisão dos métodos na reconstrução dos modelos recuperados. Os estudos de modelagem sísmica mostram que a solução mais acurada é realmente a proposta recentemente e a mais eficiente computacionalmente demonstra menor precisão. O estudo tomográfico mostra que há diferenças na reconstrução do modelo em relação ao cálculo dos tempos de trânsito resultante de cada método analisado.     
	
    \textbf{Palavras-chave}: equação eikonal; sísmica de refração; comparação analítica-numérica; tomografia sísmica.
\end{resumo}
