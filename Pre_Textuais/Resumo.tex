%========================================================================
% Modelo para elaboracao de textos academicos: TCC, dissertacoes e teses
% Elaborado pelo GISIS - Grupo de Imageamento Sismico e Inversao Sismica.
%========================================================================
\setlength{\absparsep}{18pt} % ajusta o espaçamento dos parágrafos do resumo
\begin{resumo}
    \fonteResumo

\begin{flushleft}
\MakeUppercase{\imprimirSobrenome}, \imprimirPrimeirosNomes. \textbf{\imprimirtitulo}. \imprimirtipotrabalho, \imprimirinstituicao. \imprimirlocal,  p. \pageref{LastPage}. \imprimirAnoDeDefesa.
\end{flushleft}

Segundo a ABNT NBR6028:2003, o resumo deve ressaltar o objetivo, o método, os resultados e as conclusões do documento. Algumas recomendações para escrever um bom resumo são: dar enfase aos resultados e conclusões, apresentar brevemente as perguntas respondidas, e o design experimental. O resumo deve ser sucinto e ter somente um parágrafo, escrito no tempo verbal passado. As palavras-chave devem figurar logo abaixo do resumo, antecedidas da expressão ``Palavras-chave:". Seguindo a atualização da norma realizada em 2021, as palavras-chave devem ser escritas com letra minúscula no início, exceto no caso de nomes próprios, e separadas por ponto e vírgula \cite{palavraschave}. \citeonline{Landes1969} fornece algumas recomendações para a redação de um bom resumo. Além disso a UFF possui um documento com recomendações para apresentação de trabalhos acadêmicos \cite{Abreu2012}.

    \textbf{Palavras-chave}: latex; abntex; editoração de texto.
\end{resumo}
