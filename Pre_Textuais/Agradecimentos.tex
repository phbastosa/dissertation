%========================================================================
% Modelo para elaboracao de textos academicos: TCC, dissertacoes e teses
% Elaborado pelo GISIS - Grupo de Imageamento Sismico e Inversao Sismica.
%========================================================================
\begin{agradecimentos}
    \fonteAgradecimentos 
    
    \noindent Neste elemento pré-textual, a escrita é livre. Busca-se reconhecer e expressar a gratidão à todos que, de alguma forma, fizeram parte da construção deste texto e do caminho percorrido ao longo do desenvolvimento do trabalho.
        
    \noindent Podem ser citados familiares, colegas e amigos que o apoiaram. 
        
    \noindent Além disso, é de bom tom incluir agradecimentos ao seu orientador e seu coorientador, se houver. 
    
    \noindent Deve-se lembrar da contribuição de funcionários, técnicos e professores que ajudaram no seu desenvolvimento acadêmico e profissional.
    
    \noindent Para os trabalhos que foram financiados através de bolsas de fomento à pesquisa, deve-se agradecer à agência financiadora.
    
    \noindent Deve-se mencionar também:
    
    \vspace{-.5cm}
    
    \begin{itemize}
        \item o projeto que participou;
        \item o grupo de pesquisa;
        \item a Universidade Federal Fluminense (UFF);
        \item o Programa de pós-graduação Dinâmica dos Oceanos e da Terra (DOT) e/ou ao Departamento de Geologia e Geofísica (GGO);
        \item os membros da banca examinadora;
        \item \textit{softwares} de licença acadêmica utilizados.
    \end{itemize}
\end{agradecimentos}