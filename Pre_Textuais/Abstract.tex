%========================================================================
% Modelo para elaboracao de textos academicos: TCC, dissertacoes e teses
% Elaborado pelo GISIS - Grupo de Imageamento Sismico e Inversao Sismica.
%========================================================================
\begin{resumo}[Abstract]
    \begin{otherlanguage*}{english}

\begin{flushleft}
\MakeUppercase{\imprimirSobrenome}, \imprimirPrimeirosNomes. \textbf{\imprimirTituloEmIngles}. \imprimirtipotrabalhoIngles, \imprimirinstituicaoIngles. \imprimirlocal,  p. \pageref{LastPage}. \imprimirAnoDeDefesa.
\end{flushleft}

        \fonteResumo
        The correct representation of the wave propagation phenomena is extremelly important for geological properties estimation through the seismic method. Some contributions, which involve the eikonal equation, consider only computational efficiency in their numerical resolutions. In this work, three eikonal equation solvers are analized looking for precision and performance using seismic modeling in different scenarios comparing to an analytical solution for refracted waves. Along with the precision study, another experiment is realized to verify the inaccurate effects of modeling kernel in seismic tomography based on the numerical methods used. A classical solution well known in geoscience, a new method poorly explored and an accurate solver is analized. Two inversion schemes are proposed to identify the methods inaccuracy effects in the velocity model reconstruction. Seismic modeling studies show that the most accurate solution is actually the one proposed recently and the most computationally efficient demonstrates lower accuracy. The seismic tomography shows that there are differences in the velocity model reconstruction in relation to the travel time computed resulting from each method analyzed.     
    
        \vspace{\onelineskip}
     
        \noindent 
        \textbf{Keywords}: eikonal equation; seismic refraction; analytical-numerical comparison; seismic tomography.
    \end{otherlanguage*}
\end{resumo}
