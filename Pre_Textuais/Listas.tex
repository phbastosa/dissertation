%========================================================================
% Modelo para elaboracao de textos academicos: TCC, dissertacoes e teses
% Elaborado pelo GISIS - Grupo de Imageamento Sismico e Inversao Sismica.
%========================================================================
% Lista de ilustrações %%%%%%%%
\pdfbookmark[0]{\listfigurename}{lof}
\listoffigures*
\cleardoublepage

% Lista de tabelas %%%%%%%%%%%%
\pdfbookmark[0]{\listtablename}{lot}
\listoftables*
\cleardoublepage

% Lista de abreviaturas e siglas
\begin{center}
\fonteLista Lista de abreviaturas e siglas 
\end{center}

\vspace{2cm}
\fonteNormalSize
\begin{tabular}{l l}
    UFF & Universidade Federal Fluminense \\ [1.0ex]
    GGO & Departamentode Geologia e Geofísica \\ [1.0ex]
    GISIS & Grupo de Imageamento e Inversão Sísmica \\ [1.0ex]
	WSL & \textit{Windows Subsystem for Linux} \\ [1.0ex]
	FMM & \textit{Fast Marching Method} \\ [1.0ex]
	FSM & \textit{Fast Sweeping Method} \\ [1.0ex]
	FIM & \textit{Fast Iterative Method} \\ [1.0ex]
	OBN & \textit{Ocean Bottom Nodes} \\ [1.0ex]
	CSR & \textit{Compressed Sparse Row} \\ [1.0ex]
	CSC & \textit{Compressed Sparse Column}
\end{tabular}
\cleardoublepage


% Lista de símbolos
\begin{center}
    \fonteLista Lista de símbolos
\end{center}

\vspace{2cm}
\fonteNormalSize
\begin{tabular}{l l}
    $ T $ & volume de tempos de trânsito \\ [1.0ex]
    $ S $ & volume de vagarosidade \\ [1.0ex] 
    $ v $ & velocidade de propagação da onda P \\ [1.0ex]
    $ dh $ & parâmetro de discretização espacial \\ [1.0ex]
    $ d^{obs} $ & dado observado \\ [1.0ex]
    $ d^{cal} $ & dado calculado \\ [1.0ex]
    $ \chi(m) $ & função objetivo da tomografia \\ [1.0ex]
    $ G $ & matriz de modelagem direta \\ [1.0ex]
    $ \Delta $ & operador variação \\ [1.0ex]
    $ \nabla $ & operador gradiente \\ [1.0ex]
\end{tabular}

    
