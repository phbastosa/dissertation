%========================================================================
% Modelo para elaboracao de textos academicos: TCC, dissertacoes e teses
% Elaborado pelo GISIS - Grupo de Imageamento Sismico e Inversao Sismica.
%========================================================================
\chapter{Discussões}
\label{ch:discussoes}
Uma boa prática é escrever essa seção após a escrita da Introdução, pois nela se ressalta os objetivos do trabalho e as questões que se deseja responder. A discussão pode ser feita no mesmo capítulo dos Resultados, caso preferir. Algumas boas práticas são:

\begin{itemize}
	\item Descreva brevemente as limitações do estudo mostrando o que podem ser consideradas fraquezas do experimento
	\item Discuta a partir dos tópicos mais importantes para os menos importantes
	\item Relacione os resultados com as hipóteses
\end{itemize}

Tem como objetivo interpretar e explicar os resultados. De certa forma, a seção de discussão estabelecerá uma ligação entre o que você falou na introdução, com as questões de pesquisa e hipóteses e os artigos que foram citados. Portanto, essa seção irá mostrar ao leitor como o estudo se desenvolveu a partir dos questionamentos deixados na introdução. Use a voz ativa sempre que possível. Cuidado com frases prolixas, seja conciso e escreva claramente.

Algumas questões que podem ser respondidas:
\begin{enumerate}
    \item Seus resultados fornecem respostas ao seu teste de hipótese? Se sim, como você pode interpretar essas respostas?
	\item O que você achou no estudo? Estão de acordo com o que os outros têm mostrado? Se não, eles sugerem uma explicação alternativa ou uma falha na execução do estudo?
\end{enumerate}
Dicas:
\begin{itemize}
    \item Organize a discussão de acordo com os estudos sobre os quais você apresentou os resultados. Escreva seguindo e mesma ordem apresentada na seção de resultados mostrando sua interpretação sobre os resultados encontrados. Não perca tempo escrevendo novamente os resultados já mostrados na seção anterior.
	\item Se possível, você deve fazer comparações dos seus resultados com resultados de outros autores ou estudos que você já tenha feito. Isso pode ser útil para que você encontre informações importantes em outros estudos que agregam valor a sua interpretação. Considere também como esses outros resultados podem ser combinados com os seus.
	\item Não mostre novos resultados na seção de discussão. Embora você possa utilizar novas tabelas e figuras para resumir os resultados, elas não devem conter novos resultados (dados).

\end{itemize}