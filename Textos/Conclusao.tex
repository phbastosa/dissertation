%========================================================================
% Modelo para elaboracao de textos academicos: TCC, dissertacoes e teses
% Elaborado pelo GISIS - Grupo de Imageamento Sismico e Inversao Sismica.
%========================================================================
\chapter{Conclusão}
\label{ch:conclusao}

Neste trabalho é estudado e desenvolvido três métodos numéricos, o método clássico, o \textit{Fast Iterative Method} original e a forma paralelizada do \textit{Fast Sweeping Method} com operadores modificados, que resolvem a equação eikonal de forma a verificar a eficiência computacional e a precisão da solução em relação a equações analíticas. Para isso, três esquemas foram testados, sendo o primeiro a propagação em meio homogêneo, o segundo a aplicação em um modelo simples de duas camadas e o terceiro a propagação em um modelo de geologia complexa com altos contrastes de velocidade. 

Os resultados obtidos com o estudo em modelo homogêneo mostram que o \textit{Fast Iterative Method}, mesmo em sua forma mais atualizada para melhorar a precisão, não supera a formulação mais acurada neste trabalho. Sendo assim, o método que demonstrou melhor acurácia foi o \textit{Fast Sweeping Method}, onde o valor médio e máximo do erro foram os menores na tabela apresentada. 

Em relação ao estudo de onda refratada utilizando o modelo de duas camadas variando a quantidade de amostras a partir do parâmetro de discretização, foi testado além da precisão, o custo computacional de cada método para uma propagação simples e para múltiplas propagações no estudo de reciprocidade. Os resultados mostram que os erros diminuem por consequência do refinamento do modelo de velocidade, a grandeza dos erros são maiores para o método mais eficiente computacionalmente. A formulação mais acurada, o \textit{Fast Sweeping Method}, diferencia pouco em performance do método mais eficiente, o \textit{Fast Iterative Method}, sendo que possui a melhor precisão. Os estudos de reciprocidade evidenciaram uma falha na inicialização do método mais preciso, porém o erro foi corrigido com a inicialização de todos os pontos vizinhos da fonte. Em larga escala o \textit{Fast Iterative Method} se destacou em relação à performance, porém obteve os maiores erros nesse estudo em termos de precisão.     

Os métodos numéricos não sofreram dificuldades com a aplicação no modelo de geologia complexa. Os resultados mostram um atraso intrínseco nas formulações clássica e no \textit{Fast Iterative Method}. Esse atraso acaba representando um modelo com velocidades menores do que deveriam ser, prejudicando etapas de imageamento como a tomografia e a migração utilizando os tempos de trânsito gerados pela equação eikonal. Em relação a performance dos métodos aplicados ao modelo realístico obteve-se que a formulação clássica atinge os piores patamares de performance, enquanto as formulações mais recentes se destacam. Embora a formulação mais precisa não seja a mais eficiente computacionalmente, os valores de precisão que entrega são altos em relação ao método mais eficiente sendo uma diferença apenas de 36\% entre as formulações testadas.


\section{Trabalhos futuros}

Para trabalhos futuros, a partir da modelagem mais precisa, desenvolver técnicas de imageamento e inversão sísmica para resolver problemas em alta resolução. Um exemplo seria a implementação da tomografia utilizando operadores do estado adjunto, já que a utilização da equação eikonal se torna essencial nesse tipo de problema. Com uma formulação eikonal mais precisa, os valores de velocidade recuperados são representados de forma mais fidedigna. Outra aplicação para o futuro seria a migração Kirchhoff em profundidade. Utilizando uma formulação eikonal precisa, os eventos podem ser representados de forma mais fidedigna evitando, assim erros na conversão tempo - profundidade. 
