%========================================================================
% Modelo para elaboracao de textos academicos: TCC, dissertacoes e teses
% Elaborado pelo GISIS - Grupo de Imageamento Sismico e Inversao Sismica.
%========================================================================
\chapter{Conclusão}
\label{ch:conclusao}

Neste trabalho, são analisados e desenvolvidos três métodos numéricos: o método clássico, o \textit{Fast Iterative Method} original e a versão paralelizada do \textit{Fast Sweeping Method} com operadores modificados. Esses métodos são aplicados para resolver a equação eikonal, com o objetivo de avaliar tanto a eficiência computacional quanto a precisão das soluções em comparação com soluções analíticas. Para realizar essa avaliação, foram considerados três cenários distintos: a propagação em meio homogêneo, a aplicação em um modelo simples de duas camadas e a propagação em um modelo geológico complexo com significativos contrastes de velocidade.

Os resultados obtidos na investigação do modelo homogêneo apontam para uma constatação relevante: apesar das melhorias implementadas no \textit{Fast Iterative Method} visando aprimorar a precisão, essa formulação não conseguiu superar a abordagem mais precisa desenvolvida neste estudo. Nesse cenário, o método que se destacou pela melhor acurácia foi o \textit{Fast Sweeping Method}. Os valores médio e máximo de erro apresentados na tabela correspondente reforçam essa superioridade, indicando que, ao confrontar as soluções numéricas com as expectativas analíticas, o \textit{Fast Sweeping Method} alcançou desempenho mais confiável e consistente. Essa constatação respalda a escolha desse método como uma opção mais eficaz e precisa para a resolução da equação eikonal em contextos homogêneos.

No que concerne ao estudo de onda refratada, considerando o modelo de duas camadas e variando a quantidade de amostras através do parâmetro de discretização, avaliamos não apenas a precisão, mas também o custo computacional associado a cada método, tanto para uma propagação simples quanto para múltiplas propagações no estudo de reciprocidade. Os resultados indicam que, à medida que refinamos o modelo de velocidade, os erros tendem a diminuir, mas é notável que a formulação mais eficiente computacionalmente apresenta maiores magnitudes de erro. Surpreendentemente, a formulação mais precisa, o \textit{Fast Sweeping Method}, demonstra desempenho comparável ao método mais eficiente, o \textit{Fast Iterative Method}, embora mantenha a melhor precisão. Os estudos de reciprocidade revelaram uma falha na inicialização do método mais preciso, porém esse erro foi corrigido ao inicializar todos os pontos vizinhos à fonte. Em uma escala mais ampla, o \textit{Fast Iterative Method} se destacou em termos de desempenho, mas, paradoxalmente, apresentou os maiores erros em relação à precisão nesse contexto específico.

Os métodos numéricos não enfrentaram dificuldades durante a aplicação no modelo de geologia complexa. Os resultados revelam um atraso intrínseco nas formulações clássica e no \textit{Fast Iterative Method}. Essa defasagem acaba resultando em um modelo com velocidades subestimadas, prejudicando etapas cruciais de imageamento, como a tomografia e a migração, que utilizam os tempos de trânsito gerados pela equação eikonal. No que diz respeito ao desempenho computacional dos métodos no modelo realista, observa-se que a formulação clássica alcança os níveis mais baixos de performance, ao passo que as formulações mais recentes se destacam. Apesar da formulação mais precisa não ser a mais eficiente computacionalmente, os valores de precisão que proporciona são significativamente superiores em relação ao método mais eficiente, apresentando uma diferença de apenas 26\% entre as formulações testadas.

\section{Trabalhos futuros}

Para futuras investigações, a partir da modelagem mais precisa realizada neste trabalho, é possível explorar a implementação de técnicas avançadas de imageamento e inversão sísmica, especialmente voltadas para a resolução de problemas em alta definição. Uma abordagem sugerida é o desenvolvimento de algoritmos de tomografia que empreguem operadores do estado adjunto. Nesse contexto, a utilização da equação eikonal se torna essencial, pois uma formulação precisa pode contribuir significativamente para a recuperação de valores de velocidade mais fidedignos. Além disso, outra direção promissora para pesquisas futuras é a aplicação da migração Kirchhoff em profundidade. Ao empregar uma formulação eikonal mais precisa, os eventos sísmicos podem ser representados de maneira mais acurada, minimizando potenciais erros na conversão entre tempo e profundidade. Essas abordagens não apenas aprimorariam a compreensão do subsolo, mas também teriam aplicações práticas em diversas áreas, incluindo exploração de recursos naturais e estudos geotécnicos.
