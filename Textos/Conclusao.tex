%========================================================================
% Modelo para elaboracao de textos academicos: TCC, dissertacoes e teses
% Elaborado pelo GISIS - Grupo de Imageamento Sismico e Inversao Sismica.
%========================================================================
\chapter{Conclusão}
\label{ch:conclusao}

A formulação que se mostrou mais precisa a partir dos primeiros experimentos foi o \textit{Fast Sweeping Method} com operadores modificados coletando oito pontos da vizinhança para realizar os cálculos. A junção entre duas formulações: a técnica de paralelização em placa gráfica e a aplicação de operadores mais precisos, foi crucial para viabilizar a formulação acurada em exemplos realísticos, como mostrado no experimento de modelo complexo. O exemplo com modelo homogêneo mostra que o \textit{Fast Iterative Method} de ordem superior desenvolvido recentemente peca em performance e precisão em relação à nova versão do \textit{Fast Sweeping Method} com operadores acurados e paralelizado em placas gráficas desenvolvido neste trabalho.  

Os resultados da tomografia mostraram que mesmo para modelos regionais, ou seja, de baixa resolução, a imprecisão dos métodos numéricos afeta a reconstrução do modelo de velocidade, levando em consideração as pequenas variações observadas. A inversão tomográfica não foi capaz de lidar com os altos contrastes empregados no modelo de referência por limitações de estabilidade do problema inverso, como a escolha da regularização e a discretização do modelo de inversão. 

\section{Trabalhos futuros}

Para trabalhos futuros, uma abordagem mais eficiente de seleção das ondas de primeira chegada poderia ser considerada para utilização em dados reais. A formulação da tomografia utilizando operadores do estado adjunto poderia ser implementada pois algumas vantagens em relação à tomografia clássica por trajetórias de raios são observadas em trabalhos recentes. A aplicação da tomografia em dados reais seria a próxima etapa da jornada, contudo utilizando um método de modelagem direta testado e avaliado em performance e precisão. A adição de parâmetros anisotrópicos é uma opção para representar melhor as propriedades do meio em casos reais. 

