%========================================================================
% Modelo para elaboracao de textos academicos: TCC, dissertacoes e teses
% Elaborado pelo GISIS - Grupo de Imageamento Sismico e Inversão Sismica.
%========================================================================
\chapter{Revisão Bibliográfica}
\label{ch:revisaobibliografica}

A modelagem e tomografia sísmica são técnicas essenciais na exploração de recursos naturais subterrâneos e na caracterização da subsuperfície terrestre. A tomografia sísmica é um método não invasivo que permite mapear as propriedades físicas do subsolo através da análise das ondas sísmicas geradas por fontes controladas. Um dos principais desafios na tomografia sísmica é a obtenção de informações precisas sobre as trajetórias das ondas sísmicas e suas velocidades de propagação em diferentes camadas geológicas. Nesse sentido, a equação \textit{eikonal}  é uma ferramenta fundamental para a modelagem e inversão e imageamento sísmico, pois descreve as frentes de onda e permite calcular os tempos de percurso das ondas sísmicas. Neste capítulo é explorado a equação eikonal, três resoluções numéricas e sua aplicação na tomografia de refração. Ao explorar esses tópicos, este capítulo fornecerá abrangentes conceitos utilizados na geração dos resultados deste trabalho.

\section{Modelagem sísmica}

equação analítica; limitações

equações numéricas;
	
frente de onda completa (complexidade do meio)
	
simplificação da primeira chegada (limitações do meio)

\subsection*{Equação da onda para meios acústicos}

equação completa

resolução por diferenças finitas

\subsection*{Equação analítica para ondas refratadas}

livro do kearey 

variações para interfaces como mostra o livro

\subsection*{Equação \textit{eikonal}}

partir da equação da onda em meios acústicos

eikonal e transporte

\subsection*{Método clássico}

podvin e lecomte 1991 

desenvolvimento com o algoritmo time3d

\subsection*{\textit{Fast Iterative Method}}

jeong e whitaker 2008

desenvolvimento com a implementação 

\subsection*{\textit{Fast Sweeping Method}}

original fast sweeping method

desenvolvimento com a implementação

por último mostrar noble et al 2014

operadores mais precisos



\subsection*{Comparação numérica}


\begin{figure}[H]
	\centering
	\includegraphics[width = 11cm, height = 10cm]{Imgs/RevisaoBibliografica/modelGeometry.png}
	\caption{Modelo empregado no teste de precisão e performance.}
	\label{fig:configurationNumericalComparison}
\end{figure}


\begin{figure}[H]
	\centering
	\subfloat[]{\includegraphics[width=8cm,height=3.5cm]{Imgs/RevisaoBibliografica/precision_direct.png}\label{fig:rnca}}
	\subfloat[]{\includegraphics[width=8cm,height=3.5cm]{Imgs/RevisaoBibliografica/reciprocity.png}\label{fig:rncb}}
	
	\subfloat[]{\includegraphics[width=8cm,height=1.5cm]{Imgs/RevisaoBibliografica/error_pod_direct.png}\label{fig:rncc}}
	\subfloat[]{\includegraphics[width=8cm,height=1.5cm]{Imgs/RevisaoBibliografica/error_pod_reciprocity.png}\label{fig:rncd}}
	
	\subfloat[]{\includegraphics[width=8cm,height=1.5cm]{Imgs/RevisaoBibliografica/error_fim_direct.png}\label{fig:rnce}}
	\subfloat[]{\includegraphics[width=8cm,height=1.5cm]{Imgs/RevisaoBibliografica/error_fim_reciprocity.png}\label{fig:rncf}}
	
	\subfloat[]{\includegraphics[width=8cm,height=1.5cm]{Imgs/RevisaoBibliografica/error_fsm_direct.png}\label{fig:rncg}}
	\subfloat[]{\includegraphics[width=8cm,height=1.5cm]{Imgs/RevisaoBibliografica/error_fsm_reciprocity.png}\label{fig:rnch}}
	
	\caption{Comparação de precisão entre os métodos numéricos estudados.}
	\label{fig:resultsNumericalComparison}
\end{figure}







\section{Inversão tomográfica}

Tipos de tomografia (reflexão, difração, transmissão e refração)

trabalhos do GISIS

\subsection*{Tomografia de refração}
Discretização do modelo

Raios ilustrativos

Função objetivo e linearização

Regularização 

Resolução do sistema linear iterativamente

\subsection*{Obtenção do dado observado}

tipos de picking (manual, analítico, machine learning)

formulação utilizada
 
 
