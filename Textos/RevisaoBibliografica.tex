%========================================================================
% Modelo para elaboracao de textos academicos: TCC, dissertacoes e teses
% Elaborado pelo GISIS - Grupo de Imageamento Sismico e Inversão Sismica.
%========================================================================
\chapter{Revisão Bibliográfica}
\label{ch:revisaobibliografica}

A modelagem e tomografia sísmica são técnicas essenciais na exploração de recursos naturais subterrâneos e na caracterização da subsuperfície terrestre. A tomografia sísmica é um método não invasivo que permite mapear as propriedades físicas do subsolo através da análise das ondas sísmicas geradas por fontes controladas. Um dos principais desafios na tomografia sísmica é a obtenção de informações precisas sobre as trajetórias das ondas sísmicas e suas velocidades de propagação em diferentes camadas geológicas. Nesse sentido, a equação \textit{eikonal}  é uma ferramenta fundamental para a modelagem e inversão e imageamento sísmico, pois descreve as frentes de onda e permite calcular os tempos de percurso das ondas sísmicas. Neste capítulo é explorado a equação eikonal, três resoluções numéricas e sua aplicação na tomografia de refração. Ao explorar esses tópicos, este capítulo fornecerá abrangentes conceitos utilizados na geração dos resultados deste trabalho.

\section{Modelagem sísmica}

Muitos fenômenos físicos tem suas simplificações e condições para serem simulados computacionalmente. O método sísmico parte do princípio da propagação de ondas mecânicas geradas a partir de uma fonte explosiva, sendo registradas em receptores posicionados na superfície da terra ou no fundo marinho \cite{sheriff1995exploration, rosa2010analise}. Um modelo de propriedades físicas de subsuperfície é necessário para a simulação computacional. Considerando a simplificação da equação da onda para meios acústicos e isotrópicos, onde a propriedade do meio são as velocidades da onda P, as frentes de onda podem ser geradas para realizar estudos sintéticos experimentais. Existem formatos de solução para uma equação que rege um fenômeno físico, dois deles são o caso analítica e o caso numérico. A solução analítica de um problema mostra a resposta exata do fenômeno em condições específicas previamente estabelecidas e a solução numérica resolve o problema de forma geral para diversos cenários. No caso da modelagem sísmica, as condições estabelecidas se aplicam ao modelo de velocidade, resolvendo a equação em um modelo homogêneo ou com camadas plano paralelas sem variação lateral de velocidade.  

\subsection*{Equação da onda para meios acústicos}

A equação base para a simulação sísmica utilizando a simplificação para meios acústicos pode ser formulada da seguinte maneira
\begin{equation}
	\nabla^2u(\mathbf{x}, t) - \dfrac{1}{v^2(\mathbf{x})}\dfrac{\partial^2u(\mathbf{x}, t)}{\partial t^2} = f(t),	
	\label{wave_equation}
\end{equation}
\noindent onde $u(\mathbf{x},t)$ é o campo de pressão hidrostática, $t$ é o tempo de propagação e $\mathbf{x} = (x,y,z)$ são as variáveis do sistema de coordenadas para o caso 3D. $v(\mathbf{x})$ é o modelo de velocidade, $f(t)$ é uma função que determina o formato do pulso propagado e $\nabla^2 = \partial_x + \partial_y + \partial_z$ é o operador laplaciano. \citeonline{igel2017computational} mostra algumas formas de resolver a equação da onda numericamente de forma prática e aplicada. O método das diferenças finitas é um dos métodos mais utilizados em simulações sísmicas. O princípio do método é estimar o valor da derivada numericamente utilizando a série de Taylor, definida detalhadamente no trabalho de \citeonline{de2005funccoes}. Somente derivadas de segunda ordem são necessárias para resolver a equação \ref{wave_equation} e a aproximação da derivada pode ser definida a partir da seguinte expressão
\begin{equation}
	\dfrac{\partial^2 f(x)}{\partial x^2} \approx \dfrac{f[i - dh] - 2f[i] + f[i + dh]}{dh^2},
	\label{derivada_2}
\end{equation}    
\noindent onde $f(x)$ é uma função contínua arbitrária definida no eixo $x \in \mathbb{R}$ e $f[i]$ é uma função discreta definida na mesma reta com intervalos regulares $dh$. A equação da onda para meios acústicos pode ser escrita em seu formato discreto  substituindo a equação \ref{derivada_2} na equação \ref{wave_equation}, para cada eixo $x,y,z,t$ respectivamente. O método das diferenças finitas possui certas limitações como por exemplo a dispersão e instabilidade numérica \cite{aki1980quantitative}. Seguindo os trabalhos de \citeonline{mufti1990large, bulcao2004modelagem} e \citeonline{robertsson2012numerical} as condições para contornar problemas numéricos na modelagem sísmica para meios acústicos seguem na forma a seguir
\begin{equation}
	\begin{cases}
		dh \le v_{\text{min}} \,/\, (\alpha \cdot f_{\text{max}}) \\
		dt \le dh \,/\, (\beta \cdot v_{\text{max}}),
	\end{cases}
\end{equation}   
\noindent onde $dh$ e $dt$ são os parâmetros de discretização espacial e temporal respectivamente, $v_{\text{min}}$ e $v_{\text{max}}$ são as velocidades da onda P mínima e máxima no modelo, $f_{\text{max}}$ é a frequência máxima da fonte sísmica aplicada, e $\alpha$ e $\beta$ são as quantidades de amostras para representar um comprimento de onda no espaço e no tempo respectivamente. Os valores de $\alpha$ e $\beta$ variam de acordo com a ordem do operador de diferenças finitas \cite{bulcao2004modelagem}.    

A fonte sísmica pode assumir diferentes formas dependendo de seu formato ou equacionamento. Na modelagem sísmica, a fonte é aplicada como um pulso definido no tempo discreto, onde a cada passo de tempo uma amostra do sinal é injetada, em uma posição no espaço, na resolução da equação da onda. A segunda derivada da equação gaussiana \cite{ricker1953form} pode ser usada para gerar o pulso sísmico, sendo definida da seguinte maneira
\begin{equation}
	\begin{cases}
		f_p = f_{\text{max}} / (3\sqrt{\pi}) \\
		g(t) = (1 - 2 \pi (\pi f_p t)^2) \exp(-\pi (\pi f_p t)^2),
	\end{cases}
\end{equation}
\noindent sendo $f_p$ a frequência de pico, $g(t)$ a fonte sísmica e $t$ o eixo do tempo. 

Ao contrário de dados sísmicos reais, onde o planeta absorve por completo a energia da onda, em simulação computacional, para os casos onde a equação da onda não contempla fenômenos dissipativos, existem técnicas para prevenir reflexões indesejadas causadas pelas bordas do modelo finito empregado. A condição de bordas absortivas de \citeonline{cerjan1985nonreflecting} é uma abordagem clássica para resolver o problema de bordas na modelagem sísmica, porém outras formulações mais elaboradas podem ser utilizadas, uma possibilidade é aplicar a formulação descrita no trabalho de \citeonline{chern2019reflectionless}. A condição de borda utilizada neste trabalho é regida pela equação a seguir
\begin{equation}
	w(\mathbf{x}) = exp(- p(n_b - \mathbf{x})^2),	
\end{equation}      
\noindent onde $w(\mathbf{x})$ é a função amortecimento definida somente nos pontos da borda $\mathbf{x}$, $p$ é um fator de atenuação e $n_b$ é a quantidade de amostras na borda. \citeonline{bording1921finite} compara experimentos para determinar valores otimizados para $n_b$ e $p$, em contrapartida \citeonline{gao2017comparison} mostram que a técnica esponjosa absortiva foi superada pelas novas variantes utilizando modificações da equação da onda. 

Utilizando a equação da onda em simulações sísmicas, alguns tipos de onda são registradas, que para o caso acústico escalar somente a onda P, porém existem as frentes de onda refletidas e refratadas. Nas próximas seções, somente as frentes de onda transmitidas e refratadas são exploradas pois esse tipo de onda é a ferramenta principal deste trabalho. As ondas refratadas acontecem quando o afastamento entre a fonte e o receptor são suficientemente grandes, dependendo da configuração de camadas geológicas em subsuperfície. Essas ondas refratadas comumente são chamadas de primeiras chegadas, pois se propagam em camadas de maior velocidade e assim são registradas em um tempo menor que as demais frentes de onda.          

\subsection*{Equação \textit{eikonal}}

A equação eikonal pode ser definida como uma aproximação de altas frequências para a equação da onda em meios acústicos. Essa definição pode ser expandida para outras simplificações de meio, como por exemplo o meio elástico isotrópico \cite{cerveny2003seismic}. \citeonline{rawlinson2008seismic} mostram as etapas de simplificação partindo da equação \ref{wave_equation}, assumindo uma solução periódica envolvendo tempo e frequência angular, aplicando essa solução na equação da onda, simplificando os termos e aplicando o limite da frequência angular tendendo ao infinito. Assim, a equação eikonal em três dimensões segue no formato 
\begin{equation}
	\left[\dfrac{\partial T(\mathbf{x})}{\partial x}\right]^2 + \left[\dfrac{\partial T(\mathbf{x})}{\partial y}\right]^2 + \left[\dfrac{\partial T(\mathbf{x})}{\partial z}\right]^2 = \dfrac{1}{v^2(\mathbf{x})}, 	
\end{equation} 
\noindent sendo $T(\mathbf{x})$ a função tempo de trânsito avaliada na posição $\mathbf{x} = (x,y,z)$ e $s = 1/v$ pode ser chamado de vagarosidade, o inverso da velocidade. A principal aplicação da equação eikonal utilizando o método sísmico parte da modelagem sísmica, com o trabalho pioneiro de \citeonline{vidale1988finite} onde os tempo de trânsito das primeiras chegadas são calculados. Outras aplicações são os métodos de estimativa da velocidade, como a tomografia de primeira chegada \cite{zhang1998nonlinear, sei1994gradient, tromp2005seismic, taillandier2009first},  e as técnicas de imageamento sísmico, como a migração em profundidade   \cite{gray1994kirchhoff, zhang2006refraction}.  
 
\subsection*{Equação analítica para ondas refratadas}

Apesar de existir em parametrizações variadas do modelo de velocidade, as formulações para o cálculo das primeiras chegadas possuem certas limitações \cite{kearey2002introduction}. Somente uma variação será explorada, onde o modelo de velocidades deve ser plano paralelo com camadas horizontais sem variação de velocidade lateral e a fonte e os receptores devem estar na mesma profundidade. A equação analítica para este caso pode ser escrita de forma generalizada no formato a seguir  
\begin{equation}
	\begin{cases}
		t_n = \dfrac{x}{v_n} + \displaystyle\sum_{i=1}^{n-1} \dfrac{2z_i cos(\theta_{in})}{v_i}\\
		\theta_{in} = \arcsin(v_i / v_n),
	\end{cases}
	\label{analytic_refractions}
\end{equation}
\noindent onde $t_n$ e $v_n$ é o tempo e a velocidade da $n$-ésima camada, $x$ é a distância entre a fonte e o receptor, $z_i$ é a espessura da $i$-ésima camada e $\theta_{in}$ é o ângulo crítico da refração. 

As componentes da equação \ref{analytic_refractions}, ilustradas para uma trajetória entre fonte e receptor, são apresentadas na figura \ref{fig:refracted_analytic}. Na prática, um arranjo de receptores para uma fonte, as velocidades e as espessuras das camadas são consideradas. Então para cada interface refratora, as camadas anteriores são utilizadas, que para o caso de uma interface a equação \ref{analytic_refractions} pode ser definida da seguinte forma 
\begin{equation}
	\begin{cases}
		t = \dfrac{x}{v_2} + \dfrac{2z_1cos(\theta_{12})}{v_1}\\
		\theta_{12} = \arcsin(v_1 / v_2),
	\end{cases}
	\label{analytic_one_layer_case}
\end{equation}   
\noindent e para duas interfaces, a equação \ref{analytic_refractions} se torna
\begin{equation}
	\begin{cases}
		t = \dfrac{x}{v_3} + \dfrac{2z_1cos(\theta_{13})}{v_1} + \dfrac{2z_2cos(\theta_{23})}{v_2}\\
		\theta_{13} = \arcsin(v_1 / v_3) \\ 
		\theta_{23} = \arcsin(v_2 / v_3), 
	\end{cases}
	\label{analytic_two_layer_case}
\end{equation}   
\noindent enquanto a onda direta recebe o tempo da cinemática clássica $t = x / v_1$. Com as equações abordadas acima, é possível gerar um sismograma sintético somente com tempos analíticos da primeira chegada. Para isso é necessário coletar os tempos mínimos gerados por cada interface refratora inclusive a onda direta. Essa abordagem analítica desenvolve o papel, neste trabalho, de mensurar os erros referentes aos métodos numéricos estudados.

\begin{figure}[H]
	\centering
	\includegraphics[width = 14cm, height = 8cm]{Imgs/RevisaoBibliografica/refracted_analytic.png}
	\caption{Esquema de modelagem analítica dos tempos da onda refratada. S e R são as posições da fonte e do receptor respectivamente. V e Z são as velocidades e as espessuras das camadas. $\theta$ é o angulo crítico das refrações e $i$ e $n$ se relacionam com a equação \ref{analytic_refractions}. Esquema geral com limitações entre posição de fonte e receptor, onde precisam estar na mesma profundidade, e velocidades, onde devem ser constantes lateralmente contendo camadas planas e horizontais.}
	\label{fig:refracted_analytic}
\end{figure}

\subsection*{Método clássico}

O método base utilizado para mensurar os avanços tanto em precisão quanto em eficiência computacional dos métodos numéricos que resolvem a equação eikonal foi a formulação de \citeonline{podvin1991finite}. Grandes contribuições foram publicadas em relação a localização não linear de epicentro e hipocentro de terremotos utilizando a formulação clássica, originadas com o trabalho de  \citeonline{wittlinger1993earthquake}, além das contribuições em tomografia e migração já mencionadas. A formulação de \citeonline{podvin1991finite} apresenta algumas limitações como instabilidade em camadas com alto angulo de mergulho \cite{afnimar2000finite} e discrepância no cálculo dos tempos de trânsito utilizando a técnica da reciprocidade, onde a fonte assume a posição dos receptores e vice-versa \cite{tryggvason2006travel}. \citeonline{lomax2009earthquake} hospeda um repositório no \textit{GitHub}, de livre acesso, contendo os códigos originais para o cálculo da primeira chegada.  

A resolução da equação eikonal parte do princípio de Huygens, onde cada ponto pertencente a uma frente de onda pode se comportar como uma fonte que expande outra frente de onda. Então, a solução de \citeonline{podvin1991finite} aplica sistematicamente o princípio de Huygens considerando algumas possibilidades de propagação. Cada frente de onda, como por exemplo: difrações, refrações e onda direta, é calculada de forma independente sendo atualizada no volume de tempos como a opção de menor tempo entre as demais. Especificamente, operadores de diferenças finitas para casos 1D, 2D e 3D são aplicados, e considerando a indexação da Figura \ref{fig:voxel_full}, os principais operadores serão mostrados. 
\begin{figure}[H]
	\centering
	\includegraphics[width = 7cm, height = 6cm]{Imgs/RevisaoBibliografica/voxel_full.png}
	\caption{Esquema de indexação para o cálculo dos operadores de \citeonline{podvin1991finite}. O ponto vermelho é o alvo da propagação e os demais são seus vizinhos.}
	\label{fig:voxel_full}
\end{figure}
Os operadores 1D levam em consideração somente dois pontos, porém dependendo do ângulo de incidência as propriedades de vagarosidade devem ser escolhidas minuciosamente. O primeiro operador 1D se relaciona com a onda de corpo com propagação direta no sentido $(i,j-1,k) \to (i,j,k)$, seguindo os índices mostrados na Figura \ref{fig:voxel_full}, recebe a formulação
\begin{equation}
	T_{i,j,k} = T_{i,j-1,k} + dh\, \mathbf{min}(S_{i,j-1,k},\, S_{i,j-1,k-1},\, S_{i-1,j-1,k},\, S_{i-1,j-1,k-1}),
	\label{1d_head_wave}
\end{equation}   
\noindent onde $T$ e $S$ são os volumes de tempo e de vagarosidade respectivamente e $dh$ é o espaçamento espacial do modelo. Existem mais 5 direções para se aplicar o operador da equação \ref{1d_head_wave}, totalizando 6 operadores no total. Para o caso 1D em ondas difratadas, que ocorrem nas diagonais dos cubos, os operadores nos sentidos $(i-1,j-1,k-1) \to (i,j,k)$, $(i-1,j,k-1) \to (i,j,k)$ e $(i-1,j-1,k) \to (i,j,k)$, são respectivamente do seguinte formato
\begin{equation}
	\begin{cases}
		T_{i,j,k} = T_{i-1,j-1,k-1} + \sqrt{3}\,dh\,S_{i-1,j-1,k-1} \\
		T_{i,j,k} = T_{i-1,j,k-1} + \sqrt{2}\,dh\,\mathbf{min}(S_{i-1,j-1,k-1},\,S_{i-1,j,k-1}) \\
		T_{i,j,k} = T_{i-1,j-1,k} + \sqrt{2}\,dh\,\mathbf{min}(S_{i-1,j-1,k-1},\,S_{i-1,j-1,k}).
	\end{cases}
	\label{1D_diffractions}
\end{equation}
\noindent Como algumas frentes de onda se propagam entre dois cubos, a vagarosidade mínima é estrategicamente calculada para assegurar que a frente de onda se propagará pelo meio de maior velocidade, respeitando assim o princípio de Fermat. Operadores 1D de ondas difratadas são aplicados em 20 direções ao todo. Os operadores 2D seguem uma condição de iluminação sendo formulados, utilizando os pontos $(i-1,j,k-1)$ e $(i-1,j,k)$, a seguir
\begin{equation}
	\begin{cases}
		(1) \,\, S_{ref} = \mathbf{min}(S_{i-1,j-1,k-1},\, S_{i-1,j,k-1})  \\
		(2) \,\, 0 \le (T_{i-1,j,k} - T_{i-1,j,k-1}) \le \sqrt{2}\,dh\,S_{ref} \\
		(3) \,\, T_{i,j,k} = T_{i-1,j,k} + \sqrt{(dh\,S_{ref})^2 - (T_{i-1,j,k} - T_{i-1,j,k-1})^2},	
	\end{cases}
	\label{2D_operators}
\end{equation}
\noindent sendo $(1)$ a aplicação do princípio de Fermat, $(2)$ a condição de iluminação, $(3)$ o cálculo do tempo de trânsito e $S_{ref}$ é a vagarosidade de referência em cada operador. Para esse caso, vinte e quatro operadores são aplicados nas redondezas do ponto alvo $T_{i,j,k}$. Os operadores 3D são altamente condicionais, utilizam três pontos vizinhos e para cada cubo do modelo, três operadores são aplicados. Para melhor entendimento, a Figura \ref{fig:voxel3d} mostra as direções para um cubo, dos pontos utilizados para o cálculo dos operadores 3D seguindo a simbologia do trabalho de \citeonline{podvin1991finite}. 

\begin{figure}[H]
	\centering
	
	\subfloat[]{\includegraphics[width=4.5cm,height=4cm]{Imgs/RevisaoBibliografica/3D_xz.png}}
	\subfloat[]{\includegraphics[width=4.5cm,height=4cm]{Imgs/RevisaoBibliografica/3D_yz.png}}
	\subfloat[]{\includegraphics[width=4.5cm,height=4cm]{Imgs/RevisaoBibliografica/3D_xy.png}}
		
	\caption{Esquema de pontos utilizados na aplicação dos operadores 3D de \citeonline{podvin1991finite}. (a) plano xz, (b) plano yz e (c) plano xy. Pontos em verde são a vizinhança selecionada, o ponto em contorno azul é a posição da vagarosidade de referência, e o ponto em vermelho é o alvo da propagação.}
	\label{fig:voxel3d}
\end{figure}
\noindent Então, os operadores 3D seguem o mesmo padrão para quaisquer pontos (P,Q,M,N) selecionados corretamente. As formulações completas para uma composição de pontos mostrados na Figura \ref{fig:voxel3d}, podem ser escritas da maneira a seguir
\begin{multline}
	MNP \to T_{i,j,k}: \\ 
	M \le N \,\cap\, M \le P \\
	2(P-M)^2 + (N-M)^2 \le (dh\,S_{ref})^2 \\
	2(N-M)^2 + (P-M)^2 \le (dh\,S_{ref})^2 \\
	(N-M)^2 + (P-M)^2 + (N-M)(P-M) \ge 0.5(dh\,S_{ref})^2 \\
	T_{i,j,k} = N + P - M \sqrt{(dh\,S_{ref})^2 - (N-M)^2 - (P-M)^2}, \,\,\,\,\,\,\,\,\,\,\,\,\,\,\,\,\,\,\,\,\,
	\label{MNP-R}
\end{multline}
\begin{multline}
	QNP \to T_{i,j,k}: \\	
	N \le Q \,\cap\, P \le Q \\
	(Q-N)^2 + (Q-P)^2 + (Q-N)(Q-P) \le 0.5(dh\,S_{ref})^2 \\
	T_{i,j,k} = Q + \sqrt{(dh\,S_{ref})^2 - (Q-N)^2 - (Q-P)^2}, \,\,\,\,\,\,\,\,\,\,\,\,\,\,\,\,\,\,\,\,\,\,\,\,\,\,\,\,\,\,\,\,\,   
	\label{QNP-R}
\end{multline}
\begin{multline}
	NMQ \to T_{i,j,k}: \\	
	N \ge M \,\cap\, N-M \le Q-N \\
	2(Q-N)^2 + (N-M)^2 \le (dh\,S_{ref})^2 \\
	T_{i,j,k} = Q + \sqrt{(dh\,S_{ref})^2 - (Q-N)^2 - (N-M)^2}, \,\,\,\,\,\,\,\,\,\,\,\,\,\,\,\,\,\,\,\,\,\,\,\,\,\,\,\,\,\,\,\,
	\label{NMQ-R}
\end{multline}
\begin{multline}
	PMQ \to T_{i,j,k}: \\	
	P \ge M \,\cap\, P-M \le Q-P \\
	2(Q-P)^2 + (P-M)^2 \le (dh\,S_{ref})^2 \\
	T_{i,j,k} = Q + \sqrt{(dh\,S_{ref})^2 - (Q-P)^2 - (P-M)^2}. \,\,\,\,\,\,\,\,\,\,\,\,\,\,\,\,\,\,\,\,\,\,\,\,\,\,\,\,\,\,\,\,
	\label{PMQ-R}
\end{multline}

A resolução da equação eikonal é iterativa e a cada iteração é necessário que uma frente de onda se expanda passando pelos pontos onde o tempo de trânsito é desconhecido. Inicialmente, o volume dos tempos de trânsito recebe o valor zero na fonte e um valor tendendo ao infinito nas demais posições. Os pontos vizinhos à posição da fonte são inicializados utilizando a equação da cinemática clássica $t = x / v$, considerando que cada bloco do modelo possui um valor de velocidade constante. Para posições de fonte irregulares, ou seja, posições que não pertencem a nenhum ponto da malha, o tempo da fonte é inicializado de acordo com a distância entre a fonte e o ponto da malha mais próximo. Assim, a solução iterativa de \citeonline{podvin1991finite} pode propagar o tempo inicial para as demais posições automaticamente. Após a inicialização, um volume lógico é inicializado com valores de verdadeiro e falso, sendo verdadeiro as posições que convergiram para um tempo de trânsito aceitável e falso, para posições que ainda não convergiram. O número de iterações depende da distância, em pontos de malha, entre a posição da fonte e as extremidades do modelo de velocidade, sendo a maior distância em pontos de malha que houver. Durante as iterações, todos os pontos do modelo são acessados, porém as equações são calculadas somente onde o volume lógico indica necessidade de atualização. A partir do momento que o tempo de trânsito de um certo ponto atinge a confiabilidade dada por uma tolerância, novos pontos são acionados como verdadeiros e assim, os operadores computam novos tempos em regiões ainda não exploradas até completar toda a extensão do modelo de velocidades. Para registrar os tempos de trânsito em cada posição de receptor, basta coletar o tempo do volume de tempos de trânsito e para posições irregulares, a interpolação tri-linear foi utilizada para resolver o problema.  

A solução de \citeonline{podvin1991finite} é altamente paralelizável. Um processo que pode ser paralelizado é aquele que possui problemas pequenos que podem ser resolvidos de forma independente e assim simultaneamente. O algoritmo desenvolvido pelo autor deste trabalho, para resolver a equação eikonal utilizando os operadores de \citeonline{podvin1991finite}, pode ser encontrado na plataforma de compartilhamento de código aberto GitHub.  

\subsection*{\textit{Fast Iterative Method}}

Original de \citeonline{jeong2008fast}, o \textit{Fast Iterative Method} (FIM) é uma metodologia que utiliza uma frente de onda variável que se adapta ao modelo de velocidades, tornando assim uma frente de onda mais realística. FIM é uma técnica de resolução baseada nas formulações de frente de onda expansivas como o \textit{Fast Marching Method} (FMM). O FMM, formulado por \citeonline{sethian1999fast}, possui grandes aplicabilidades em geociências como os trabalhos de \citeonline{sethian19993, rawlinson2004multiple, lelievre2011computing, jiang2021fast} e até a presente data deste trabalho ainda é explorado com diferentes implementações \cite{white2020pykonal, chenpyekfmm2023}. Por outro lado o FIM, ainda pouco estudado na área de geociências, possui vantagens computacionais em relação ao FMM, onde em cada iteração se resolve uma ordenação com a estrutura de dados \textit{heap} para atualizar a frente de onda irregular (\textit{heap sort}). O FIM não depende de estruturas de dados e cada ponto pertencente à frente de onda irregular é resolvido simultaneamente. A estratégia do FMM pode ser eficiente em computação sequencial, porém torna-se ineficiente utilizando paralelismo de larga escala até o momento da publicação de \citeonline{jeong2008fast}. Alguns anos depois, \citeonline{yang2017highly} resolvem a aplicabilidade do FMM, porém essa solução não foi considerada neste estudo. A ênfase neste trabalho foi a experimentação da formulação de \citeonline{jeong2008fast}, ainda pouco explorada. 

Outras contribuições como a implementação do FIM em diferentes domínios \cite{fu2011fast, fu2013fast} e melhoria da eficiência computacional \cite{dang2014fast, hong2016multi, hong2022mg} foram desenvolvidas. A descoberta da imprecisão dos tempos de trânsito utilizando a formulação de \citeonline{jeong2008fast} foi primordial para a aplicação do método em estudos de modelagem e imageamento sísmico. \citeonline{cai2023improved} melhorou a precisão do FIM aumentando a quantidade de pontos vizinhos para o cálculo dos tempos de trânsito sem diminuição significativa da eficiência computacional, porém essa contribuição não foi considerada neste estudo.  
       







\subsection*{\textit{Fast Sweeping Method}}

original fast sweeping method

\cite{zhao2005fast}

\cite{zhao2007parallel}

\cite{detrixhe2013parallel}

\cite{noble2014accurate}


\subsection*{Comparação numérica}


\begin{figure}[H]
	\centering
	\includegraphics[width = 11cm, height = 10cm]{Imgs/RevisaoBibliografica/modelGeometry.png}
	\caption{Modelo empregado no teste de precisão e performance. (a) Plano XY ilustrando a geometria de aquisição com o arranjo de receptores circulares possuindo somente um tiro central. Isócronas mapeando o comportamento dos tempos de trânsito são mostradas. (b) Perfil de velocidades delimitando a posição da interface. (c) e (d) são as projeções dos cortes em planos XZ e YZ em relação à posição da fonte.}
	\label{fig:configurationNumericalComparison}
\end{figure}


\begin{figure}[H]
	\centering
	\subfloat[]{\includegraphics[width=8cm,height=3.5cm]{Imgs/RevisaoBibliografica/precision_direct.png}\label{fig:rnca}}
	\subfloat[]{\includegraphics[width=8cm,height=3.5cm]{Imgs/RevisaoBibliografica/reciprocity.png}\label{fig:rncb}}
	
	\subfloat[]{\includegraphics[width=8cm,height=1.5cm]{Imgs/RevisaoBibliografica/error_pod_direct.png}\label{fig:rncc}}
	\subfloat[]{\includegraphics[width=8cm,height=1.5cm]{Imgs/RevisaoBibliografica/error_pod_reciprocity.png}\label{fig:rncd}}
	
	\subfloat[]{\includegraphics[width=8cm,height=1.5cm]{Imgs/RevisaoBibliografica/error_fim_direct.png}\label{fig:rnce}}
	\subfloat[]{\includegraphics[width=8cm,height=1.5cm]{Imgs/RevisaoBibliografica/error_fim_reciprocity.png}\label{fig:rncf}}
	
	\subfloat[]{\includegraphics[width=8cm,height=1.5cm]{Imgs/RevisaoBibliografica/error_fsm_direct.png}\label{fig:rncg}}
	\subfloat[]{\includegraphics[width=8cm,height=1.5cm]{Imgs/RevisaoBibliografica/error_fsm_reciprocity.png}\label{fig:rnch}}
	
	\caption{Comparação de precisão entre os métodos numéricos estudados. (a) Mapeamento de todas as chegadas para os métodos numéricos testados para os espaçamentos estudados. (b) Estudo de reciprocidade utilizando o espaçamento de 25 m. (c) Escala do erro para as chegadas em diferentes espaçamentos e (d) tempo direto e recíproco utilizando a formulação de \citeonline{podvin1991finite}. (e) Escala do erro e (f) estudo de reciprocidade para a formulação de \citeonline{jeong2008fast}. (g) Escala do erro e (h) estudo de reciprocidade para a formulação de \citeonline{noble2014accurate}.}
	\label{fig:resultsNumericalComparison}
\end{figure}







\section{Inversão tomográfica}

Tipos de tomografia (reflexão, difração, transmissão e refração)







trabalhos do GISIS

\cite{santos2012tomography}
\cite{bulhoes2021efeitos}



teoria de inversão

Função objetivo e linearização

Regularização 

\cite{seo2012nonlinear}
\cite{sain2023active}



least squares conjugate gradient

\cite{saad2003iterative}




\subsection*{Tomografia de refração}

Discretização do modelo

Raios ilustrativos



\subsection*{Obtenção do dado observado}



tipos de picking (manual, analítico, machine learning)

formulação utilizada

\cite{pan2019automatic} 
\cite{qin2021first} 
