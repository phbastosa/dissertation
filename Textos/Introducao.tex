%========================================================================
% Modelo para elaboracao de textos academicos: TCC, dissertacoes e teses
% Elaborado pelo GISIS - Grupo de Imageamento Sismico e Inversao Sismica.
%========================================================================
\chapter{Introdução}
\label{ch:introducao}

Problemas físicos podem ser formulados matematicamente e estimados a partir de métodos numéricos. A precisão das soluções estimadas é essencial para garantir uma boa representatividade do fenômeno físico. O método sísmico de exploração parte da deflagração artificial de ondas mecânicas e a amplitude da oscilação dessas ondas é registrada por sensores na superfície, em aquisições \textit{onshore} e \textit{offshore}, ou no fundo marinho, em aquisições \textit{offshore} permanentes. Para simular numericamente a propagação de ondas, simplificações do meio são necessárias. Duas delas são a equação da onda para meios acústicos de densidade constante ou a equação do raio cinemático. Em estudos de modelagem sísmica para meios isotrópicos, a simplificação por raios se propagando perpendicularmente à frente de onda é utilizada gerando resultados rápidos pois a solução das equações do raio é barata computacionalmente \cite{zhang1998nonlinear}. Contudo, com o avanço computacional, simplificações robustas da forma de onda se tornam viáveis como a simplificação por equação eikonal. \citeonline{hogan2007ray} apresentam as vantagens da aplicação entre a equação do raio e a equação eikonal em meios de altos contrastes de velocidade. Então, a principal motivação deste trabalho é realizar a propagação de ondas utilizando a equação eikonal de forma mais precisa possível em modelos de alta resolução para favorecer etapas futuras no processo de imageamento sísmico como a tomografia de primeira chegada e a migração sísmica em profundidade.

Atualmente há crescente necessidade de estimativa de propriedades elásticas a partir dos levantamentos sísmicos permanentes utilizando \textit{Ocean Bottom Nodes}. \citeonline{lopez2020refraction} utilizaram ondas refratadas para monitoramento de reservatório aplicando uma geometria vantajosa com largos afastamentos entre fonte e receptor em formato circular. Outras linhas de pesquisa utilizando esse mesmo tipo de geometria de aquisição se consolidaram como o trabalho de \citeonline{costa2020understanding}, que utiliza a equação da onda para verificar a iluminação no reservatório, e o de \citeonline{da2022klein}, que verifica as implicações de um meio com densidade variável no dado sísmico. As análises anteriores foram realizadas ou por equação do raio, onde o modelo necessita ser suavizado para o funcionamento estável do método, ou por equação da onda completa, onde o custo computacional é elevado. A equação eikonal é a forma intermediária para calcular a trajetória cinemática das ondas refratadas, com baixo custo computacional em relação à equação da onda e podendo ser aplicada em modelos de altos contrastes de velocidade. 

Com base na contextualização, o objetivo desde trabalho é mostrar a precisão e performance computacional de três métodos numéricos que resolvem a equação eikonal. O método de \citeonline{podvin1991finite} foi escolhido por estar consolidado na literatura, a formulação de \citeonline{jeong2008fast} é pouco explorada em geociências, porém, é desenvolvida para resolver os problemas de forma eficiente computacionalmente utilizando programação paralela e a formulação de \citeonline{noble2014accurate} é desenvolvida para ser precisa, então, este trabalho se propõe a avaliar essas formulações em precisão e performance. O \textit{Fast Iterative Method} \cite{jeong2008fast} foi desenvolvido empiricamente para substituir o \textit{Fast Marching Method} \cite{sethian19993} já consolidado. O \textit{Fast Sweeping Method}, originalmente de \citeonline{zhao2005fast}, recebe novos operadores precisos com a formulação de \citeonline{noble2014accurate} e pode ser paralelizado em placas gráficas com a formulação de \citeonline{detrixhe2013parallel}. No decorrer dos experimentos deste trabalho uma variação do \textit{Fast Iterative Method} é apresentada em \citeonline{cai2023improved}, onde o autor aumenta a ordem dos operadores de diferenças finitas para melhorar a precisão da formulação. Para averiguar a inovação, o código disponibilizado por \citeonline{cai2023improved} é utilizado em um experimento idêntico ao original detalhando assim o tempo de execução e os erros em relação a equação analítica em meio homogêneo. Os demais experimentos neste trabalho apresentam somente os métodos desenvolvidos pelo autor dessa dissertação apresentados anteriormente.  
 
No capítulo de revisão bibliográfica, os conceitos base são apresentados como a modelagem sísmica para meios acústicos de densidade constante, a equação eikonal, uma opção de equação analítica para validar os resultados da onda refratada e os métodos numéricos explorados. No capítulo de metodologia, os experimentos numéricos são apresentados, validando os tempos de trânsito gerados pela solução da equação eikonal em comparação com a equação analítica em modelos simples e complexos. No capítulo de resultados e discussões, as tabelas com os tempos de execução dos métodos são apresentadas juntamente com os resultados de precisão seguidos das respectivas discussões. Por fim, no capítulo de conclusão, o melhor método numérico estudado apontando os principais resultados e os trabalhos futuros são mencionados.

