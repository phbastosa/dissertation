%========================================================================
% Modelo para elaboracao de textos academicos: TCC, dissertacoes e teses
% Elaborado pelo GISIS - Grupo de Imageamento Sismico e Inversao Sismica.
%========================================================================
\chapter{Introdução}
\label{ch:introducao}

A seção de Introdução, assim como o resumo (ou \textit{abstract}), é considerada a porta de entrada para que o leitor se interesse pelo texto. Veja a seguir algumas funções e objetivos da Introdução:

\begin{itemize}
    \item Apresentar o tópico ou contexto que será discutido no artigo ou monografia. Para isso, podem ser citados os estudos mais importantes da área e incluir seu ponto de vista sobre o problema.
    
    \item Descrever a proposta do trabalho. Essa descrição pode ser feita por meio de hipóteses, perguntas, ou usando o problema que se pretende abordar.
    
    \item Explicar brevemente o problema que tentará solucionar ou até mesmo a abordagem utilizada, e, se possível, mencionar alguns resultados do seu estudo.
    
    \item Incluir no último parágrafo da Introdução um parágrafo sobre a estrutura completa do documento, mostrando o que será descrito em cada uma das seções seguintes.
\end{itemize}

    Uma maneira de capturar o interesse do leitor é estruturar o texto como um funil. São descritos os aspectos gerais, mostrando o contexto no qual o trabalho está inserido. Em seguida, foca-se em um tópico mais específico (ex. contexto cientifico) até chegar na proposta do trabalho e razão de sua execução.

    No início da escrita da introdução, deve-se apontar claramente qual a área de interesse. Para isso, pode-se selecionar algumas palavras-chave do título do documento para auxiliar na elaboração das primeiras sentenças da introdução. Esta estratégia garante que o assunto principal do documento seja identificado sem perder o foco. 
    
    O contexto do estudo pode ser estabelecido usando uma revisão breve e equilibrada dos artigos existentes naquela área. É interessante que seja apresentado para o leitor o que se sabe sobre o problema antes de entrar em detalhes de experimentos ou estudos. Essa breve revisão pode ser feita considerando os artigos chaves a respeito do tópico abordado no estudo. A profundidade em que se deve reportar esses artigos não é uma tarefa fácil, mas com a prática e leitura de outros artigos isso se tornará natural. 
    
    Conduzir o leitor do contexto mais geral para o mais especifico até chegar na sua proposta, tudo isso feito de forma suave, é um fator determinante para que bons resultados sejam obtidos.
    
    Os artigos que servirão de referência para o estudo do problema devem ser encontrados em revistas/periódicos nacionais e internacionais. Isto irá fundamentar seu texto, agregando confiabilidade nas informações apresentadas.
    
    Exemplos:
    \begin{enumerate}
        \item http://www.periodicos.capes.gov.br
        \item https://scholar.google.com.br
        \item http://bdtd.ibict.br/vufind
        \item https://ciencia.science.gov
        \item ScienceResearch.com
        \item https://www.scielo.br
    \end{enumerate}
    
    Artigos de periódicos são uma boa escolha, pois a maioria deles apresenta uma melhor qualidade, além de tratar de tópicos originais em sua maioria das vezes. O fato de considerar esses artigos não impede que você leia estudos publicados em anais de conferência. Estes são importantes para que você tenha uma base sobre certo tópico que irá escrever. Quando iniciar a escrita dessa porção da introdução, procure citar os artigos de revista que mostram resultados relevantes na sua área de pesquisa (https://www.scimagojr.com/). Revisões da literatura já publicadas são bem úteis, uma vez que resume toda a pesquisa feita sobre aquele tópico considerando um intervalo de tempo.
    
    Tenha certeza de que escreveu de forma clara a sua proposta e/ou hipótese que irá investigar. Pode-se escrever a proposta de forma suave acompanhando o desenvolvimento normal do parágrafo ou usando sentenças como:

    \begin{enumerate}
        \item O objetivo deste estudo é... ou
        \item Três diferentes mecanismos serão investigados para explicar o ...
    \end{enumerate}
    
    Na maioria das vezes, essas sentenças são escritas perto do final da introdução, geralmente no final do parágrafo. %Escreva de forma compreensível a justificativa da solução do problema proposto.
    %a razão pela qual a sua pesquisa resolve (ou não resolve) o problema estudado. 
    %Essa informação não deve seguir a sentença colocada anteriormente sobre a proposta do trabalho. 
    
    Tente responder perguntas como: ``Por que determinado tipo de método de pesquisa foi escolhido?", ``Quais as métricas que foram utilizadas no estudo?"\ É importante salientar que as técnicas e protocolos seguidos pelo estudo não precisam ser detalhados neste parágrafo. lsto será de responsabilidade da próxima seção relacionada à metodologia.
    
    Mais recomendações para a redação de uma boa introdução podem ser encontradas em \citeonline{Claerbout1959}.
    
    \section*{Dicas sobre a formatação do texto}
    
    Palavras estrangeiras devem ser escritas em \textit{itálico}.

    Há certas convenções que não são unânimes, portanto, se não houver uma regra, opte por um caminho e mantenha o padrão ao longo de todo o texto.
    
    \section*{Citações}
    Para incluir corretamente as citações das referências bibliográficas é necessário um arquivo com as bibliografias (*.bib). Esse arquivo pode ser feito com um bloco de notas digitando todas as informações necessárias. Porém esses arquivos podem ser encontrados facilmente com uma pesquisa na internet ou utilizar o Software Mendeley para organizar e gerar o arquivo *.bib para a referência desejada. 
    
    A seguir estão os exemplos de inclusão de referências no texto:
    \begin{itemize}
        \item \cite{yilmaz2001} - esse formato é utilizado no final da frase.
        \item \citeonline{yilmaz2001} -  esse formato é utilizado para incluir a referência como parte do texto.
    \end{itemize}
    
    \subsection*{Citações diretas e indiretas}
    
    Na citação indireta, utiliza-se no trabalho ideias do texto original mas expressas com as suas próprias palavras.
    
    ``A citação direta é feita com as palavras do próprio autor do texto original"\ \cite{citacoes}. Nesse caso é necessário prestar especial atenção à formatação. Uma citação direta curta pode ser simplesmente escrita entre aspas utilizando a mesma fonte do texto principal, como foi feito no início desse parágrafo. Uma citação direta longa, de mais de três linhas, deve seguir uma formatação diferente que, no \LaTeX, pode ser feita utilizando o ambiente:
    \begin{verbatim}
        \begin{citacao}
            ...
        \end{citacao}
    \end{verbatim}
    
    Um exemplo de citação direta longa pode ser visto a seguir:
    \begin{citacao}
    Os levantamentos geofísicos medem a variação de algumas grandezas físicas com respeito tanto à posição quanto ao tempo. A grandeza pode ser, por exemplo, a intensidade do campo magnético da Terra ao longo de um perfil cortando uma intrusão ígnea. Pode ser o movimento do terreno como uma função de tempo associada à passagem de ondas sísmicas \cite{kearey2009}.
    \end{citacao}
    
    A referência também poderia ter sido incluída antes da citação, ao invés de ao final. Por exemplo:
    
    De acordo com \citeonline{kearey2009},
    \begin{citacao}
    os levantamentos geofísicos medem a variação de algumas grandezas físicas com respeito tanto à posição quanto ao tempo. A grandeza pode ser, por exemplo, a intensidade do campo magnético da Terra ao longo de um perfil cortando uma intrusão ígnea. Pode ser o movimento do terreno como uma função de tempo associada à passagem de ondas sísmicas.
    \end{citacao}
    
    \subsection*{Citações de citações (\textit{apud})}
    
    De forma geral, citação de citação é quando se insere no texto a citação de um autor que foi encontrada em outra obra. 
    O ideal é consultar a obra original mas, no caso da citação ser relevante e a obra original não estar acessível, pode-se usar a citação da citação (com moderação).
    
    Alguns exemplos de textos usando o \emph{apud} são mostrados a seguir:
    
    Segundo \apudonline{martins2011}{oliveira2013}, um algoritmo é uma sequência de instruções que podem ser executadas de modo a atingir um determinado objetivo, associado à solução de um problema.
    
    O conceito de algoritmo é fundamental em informática, considerando que os computadores só fazem aquilo que foi mandado, e não necessariamente o que se deseja que façam \apud{martins2011}{oliveira2013}.
    
    \begin{citacao}
``...a alfabetização é mais que o simples domínio psicológico e mecânico de técnicas de escrever e de ler. É o domínio destas técnicas em termos conscientes. (...) Implica numa autoformação de que possa resultar uma postura interferente do homem sobre seu contexto."\ \apud{freire1980}{sasseron2011}
\end{citacao}
    

